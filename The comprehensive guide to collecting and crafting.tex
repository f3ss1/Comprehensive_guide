% !TEX TS-program = xelatex

%
%Не забыть:
%--------------------------------------
%Вставить колонтитулы, поменять название на титульнике



%--------------------------------------

\documentclass[a4paper, 9pt, twocolumn]{book} 
\raggedbottom

%--------------------------------------
%Part for XeLaTex for Russian language (delete in case using pdfLatex)
%--------------------------------------
% Languages.
\usepackage{polyglossia}
\RequirePackage{polyglossia}
\setdefaultlanguage{russian}
\setotherlanguage{english}

% Fonts.
\usepackage{fontspec}[Path=/Users/fess/Library/Fonts/]
\RequirePackage{fontspec}
\newfontfamily\russianfont[Ligatures=TeX]{Bookmania}
\newfontfamily\englishfont[Ligatures=TeX]{Bookmania}

%--------------------------------------
%Russian-specific packages
%--------------------------------------
%\usepackage[T2A]{fontenc}
%\usepackage[utf8]{inputenc}
%\usepackage[english,russian]{babel}
%\usepackage[intlimits]{amsmath}
%\usepackage{esint}
%--------------------------------------
%Hyphenation rules
%--------------------------------------
\usepackage{hyphenat}
\hyphenation{ма-те-ма-ти-ка вос-ста-нав-ли-ва-ть}
\hyphenation{предметов}
%--------------------------------------
%Packages
%--------------------------------------
\usepackage{caption}
\usepackage{amsmath}
\usepackage{amssymb}
\usepackage{amsfonts}
\usepackage{amsthm}
\usepackage{anyfontsize}
\usepackage{latexsym}
\usepackage{mathtools}
\usepackage{etoolbox}%Булевые операторы
\usepackage{extsizes}%Выставление произвольного шрифта в \documentclass
\usepackage[T1]{fontenc}
\usepackage{geometry}%Разметка листа
\usepackage{indentfirst}
\usepackage{wrapfig}%Создание обтекаемых текстом объектов
\usepackage{fancyhdr}%Создание колонтитулов
\usepackage{setspace}%Настройка интерлиньяжа
\usepackage{lastpage}%Вывод номера последней страницы в документе, \lastpage
\usepackage{soul}%Изменение параметров начертания
\usepackage{sectsty}
\usepackage{hyperref}%Две строчки с настройкой гиперссылок внутри получаеммого
\usepackage[usenames,dvipsnames,svgnames,table,rgb]{xcolor}% pdf-документа
\usepackage{multicol}%Позволяет писать текст в несколько колонок
\usepackage{cite}%Работа с библиографией
\usepackage{subfigure}% Человеческая вставка нескольких картинок
\usepackage{tikz}%Рисование рисунков
\usepackage{xcolor}%Возможность устанавливать цвет шрифта
\usepackage{float}% Возможность ставить H в положениях картинки
\usepackage{tabularx}
\newcolumntype{C}{>{\centering\arraybackslash}X}
\newcolumntype{Q}{>{\hsize=.35\hsize}X}
\newcolumntype{W}{>{\hsize=.35\hsize}C}
\newcolumntype{E}{>{\hsize=.35\hsize}C}
\newcolumntype{R}{>{\hsize=.2\hsize}C}
\newcolumntype{T}{>{\hsize=.2\hsize}X}

\usepackage[most]{tcolorbox}
\usepackage{lipsum}
\usepackage{titlesec}% Дополнительные уровни section
% Настройка titlesec:
\setcounter{secnumdepth}{0}
\setcounter{tocdepth}{1}
\titleformat*{\subsection}{\Large\bfseries}
% Настройка оглавления в 2 колонки
\usepackage[toc]{multitoc}
\renewcommand*{\multicolumntoc}{2}
%Настройка цветов секций и т.п.

% Новые цвета

\definecolor{sectioncolor}{RGB}{81, 28, 18}
\definecolor{linecolor}{RGB}{197, 173, 115}
\definecolor{altertable}{RGB}{230, 207, 191}
\definecolor{framecolor}{RGB}{250, 247, 235}




\subsectionfont{\color{sectioncolor}}
\subsubsectionfont{\color{sectioncolor}}


% Для картинок Моти
\usepackage{misccorr}
\usepackage{lscape}
\usepackage{cmap}



\usepackage{graphicx}
\graphicspath{{Pictures/}}
\DeclareGraphicsExtensions{.pdf,.png,.jpg}

\usepackage{eso-pic}

%\usepackage{background}
%\backgroundsetup{scale=1.2, angle=0, firstpage=true, opacity=1, contents={\begin{tikzpicture}[remember picture,overlay]
%		\node at([yshift=0pt, xshift=0pt]current page.center)
%		{\includegraphics{Title}};
%		\end{tikzpicture}}}

%----------------------------------------
%Список окружений
%----------------------------------------
\newenvironment {theor}[2]
{\smallskip \par \textbf{#1.} \textit{#2}  \par $\blacktriangleleft$}
{\flushright{$\blacktriangleright$} \medskip \par} %лемма/теорема с доказательством

\newenvironment {proofn}
{\par $\blacktriangleleft$}
{$\blacktriangleright$ \par} %доказательство

\newenvironment {dndtable}
{
		\bigskip
		\centering
		\begin{tcolorbox}
			[enhanced, 
			sharp corners,
			colback=framecolor, 
			boxrule = 0pt, 
			overlay={
				\begin{tcbinvclipframe}
					\path[fill=black] ([xshift=10pt,yshift=7pt]frame.north west) --
					(frame.north west) --
					(frame.north east) --
					([xshift=-10pt, yshift=7pt]frame.north east) --
					([xshift=-10pt, yshift=2pt]frame.north east) --
					([xshift=10pt,yshift=2pt]frame.north west) -- cycle;
					
					\path[fill=black] ([xshift=10pt,yshift=-7pt]frame.south west) --
					(frame.south west) --
					(frame.south east) --
					([xshift=-10pt, yshift=-7pt]frame.south east) --
					([xshift=-10pt, yshift=-2pt]frame.south east) --
					([xshift=10pt,yshift=-2pt]frame.south west) -- cycle;
				\end{tcbinvclipframe}
			}
			]}
{\end{tcolorbox}}
%----------------------------------------
%Список команд
%----------------------------------------
\newcommand{\grad}
{\mathop{\mathrm{grad}}\nolimits\,} %градиент

\newcommand{\diver}
{\mathop{\mathrm{div}}\nolimits\,} %дивергенция

\newcommand{\rot}
{\ensuremath{\mathrm{rot}}\,}

\newcommand{\Def}[1]
{\underline{\textbf{#1}}} %определение

\newcommand{\RN}[1]
{\MakeUppercase{\romannumeral #1}} %римские цифры

\newcommand {\theornp}[2]
{\textbf{#1.} \textit{ #2} \par} %Написание леммы/теоремы без доказательства

\newcommand{\qrq}
{\ensuremath{\quad \Rightarrow \quad}} %Человеческий знак следствия

\newcommand{\qlrq}
{\ensuremath{\quad \Leftrightarrow \quad}} %Человеческий знак равносильности

\renewcommand{\phi}{\varphi} %Нормальный знак фи

\newcommand{\me}
{\ensuremath{\mathbb{E}}}

\newcommand{\md}
{\ensuremath{\mathbb{D}}}

\newcommand{\newsubsection}[1]{
	{\subsection{#1}}
	\vskip -0.45cm
	{
		\noindent
		\color{linecolor}
		\rule{0.98\linewidth}{1.8pt}
	}
}

\newcommand{\partc}[2][]{{
		\bigskip
		\noindent
		\hspace{-0.25cm}
		\fontsize{11pt}{13.2}
		\color{sectioncolor}
		\textbf{#2}}
	
	{
		\ifx\relax#1\relax
		\else
		\noindent
		\normalcolor
		\textbf{#1}}
	\bigskip
	\fi
}


%\renewcommand{\vec}{\overline}




%----------------------------------------
%Разметка листа
%----------------------------------------
\geometry{top = 3cm}
\geometry{bottom = 2cm}
\geometry{left = 1.5cm}
\geometry{right = 1.5cm}
%----------------------------------------
%Колонтитулы
%----------------------------------------
\pagestyle{fancy}%Создание колонтитулов
\fancyhead{}
%\fancyfoot{}
\fancyhead[R]{\textsc{Уравнения Лондонов. Кинетическая индуктивность сверхпроводников.}}%Вставить колонтитул сюда
%----------------------------------------
%Интерлиньяж (расстояния между строчками)
%----------------------------------------
%\onehalfspacing -- интерлиньяж 1.5
%\doublespacing -- интерлиньяж 2
%----------------------------------------
%Настройка гиперссылок
%----------------------------------------
\hypersetup{				% Гиперссылки
	unicode=true,           % русские буквы в раздела PDF
	pdftitle={Заголовок},   % Заголовок
	pdfauthor={Автор},      % Автор
	pdfsubject={Тема},      % Тема
	pdfcreator={Создатель}, % Создатель
	pdfproducer={Производитель}, % Производитель
	pdfkeywords={keyword1} {key2} {key3}, % Ключевые слова
	colorlinks=true,       	% false: ссылки в рамках; true: цветные ссылки
	linkcolor=blue,          % внутренние ссылки
	citecolor=blue,        % на библиографию
	filecolor=magenta,      % на файлы
	urlcolor=red           % на URL
}
%----------------------------------------
%Работа с библиографией (как бич)
%----------------------------------------
%\renewcommand{\refname}{Список литературы}%Изменение названия списка литературы для article
%\renewcommand{\bibname}{Список литературы}%Изменение названия списка литературы для book и report
%----------------------------------------



\begin{document}
	\begin{titlepage}
		\AddToShipoutPictureBG*{%
			\put(0,0){\includegraphics[scale=1.2]{Title}}%
		}
		\begin{center}
			\begin{figure}[H]
				\centering
				\includegraphics[scale=0.2]{DnD_logo}
			\end{figure}
			{\fontsize{30pt}{0pt}\textcolor{white}{\textbf{Всеобъемлющее руководство по сбору и созданию предметов}}}
			
			\begin{figure}[H]
				\centering
				\includegraphics[scale=0.2]{Arrow}
			\end{figure}			
			
				\vskip 13cm
				
			{\Large{\textcolor{white}{Руководство по системе сбора и создания предметов в одной из лучших ролевых игр мира}}}
			
		\end{center}
	\end{titlepage}

	\AddToShipoutPictureBG{%
		\put(0,0){\includegraphics[scale=1.2]{Base}}%
	}

	\titleformat{\chapter}
	{\normalfont}
	{}
	{8pt}
	{\fontsize{1.113cm}{5}\bfseries\color{sectioncolor}\filcenter}

	\tableofcontents
	
	\titleformat{\chapter}
	{}
	{\fontsize{1.113cm}{5}\bfseries\color{sectioncolor}\filcenter Часть \thechapter:}
	{8pt}
	{\fontsize{1.113cm}{5}\bfseries\color{sectioncolor}\filright}
	
	\titleformat{\section}
	{\normalfont}
	{}
	{8pt}
	{\fontsize{18pt}{5}\selectfont\color{sectioncolor} \textbf}
	
	\titleformat{\subsection}
	{\normalfont}
	{}
	{8pt}
	{\fontsize{13pt}{5}\selectfont\color{sectioncolor} \textbf}
	[\vskip -0.25cm
	{
		\noindent
		\color{linecolor}
		\rule{0.98\linewidth}{1.8pt}
		\vskip -0.2cm
	}]

	\titleformat{\subsubsection}
	{\normalfont}
	{}
	{8pt}
	{\fontsize{11pt}{5}\selectfont\color{sectioncolor} \textbf}
	[\vskip -0.2cm]
	
	\titleformat{\paragraph}
	{\normalfont}
	{}
	{1em}
	{\textbf}
	
	\titleformat{\partc}
	{\normalfont}
	{}
	{1em}
	{\fontsize{10pt}{5} \selectfont\color{sectioncolor} }

	
	\chapter{Сбор материалов}

	%\begin{tikzpicture}
	%\centering
	%%\path[use as bounding box] (0, 0) rectangle +(0.1,0.1);
	%\draw[black, line width=0.5pt] (0.2364, -0.913) -- 
	%(0.07, -0.7466) --
	%(0.07, -0.5649) --
	%(0.07, -0.5649) arc (265:286:0.4577) --
	%(0.2364, -0.536) -- 
	%(0.2364, -0.536) arc (180:90:0.2996) --
	%(0.5485, -0.2364) --
	%(0.5485, -0.2364) arc (344:365:0.4577) -- 
	%(9.3066, -0.07);
	%\end{tikzpicture}
	
	%\begin{tcolorbox}[enhanced, interior hidden, boxrule = 0pt, frame code={
				%\path [tcb fill frame, fill=white] ([yshift=7mm]frame.south west) -- ([yshift=-7mm]frame.north west) -- ([yshift=-7mm]frame.north west) arc (300:330:20mm) -- ([xshift=-7mm]frame.north east) -- ([xshift=-7mm]frame.north east) arc (210:240:20mm) -- ([yshift=7mm]frame.south east) -- ([yshift=7mm]frame.south east) arc (120:150:20mm) -- ([xshift=7mm] frame.south west) -- ([xshift=7mm] frame.south west) arc (30:58:20mm) -- cycle;},
	%	]
	%	\lipsum[2]
	%\end{tcolorbox}
	
	

	
	
	\section{Травничество}
	
	Ремесло поиска, сбора и определения растений и ингредиентов для последующего их использования в алхимических субстанциях. Пока игроки путешествуют по миру, они могут захотеть взять с собой какие-то местные растения. Травничество используется для сбора таких вещей как лепестки, семена, корни, грибы, листья и много другое.
	
	Чтобы собирать ингредиенты, необходим набор травника, в котором есть ножницы и прочие инструменты для отделения необходимых частей растений. Для последующего хранения или продажи полученных ресурсов используются флаконы, мешочки и прочие контейнеры.
	
	Приведенное ниже руководство покажет вам три основных функции Травничества.
	
	\subsection{Поиск трав}
	
	Вы можете провести одну \textbf{проверку Мудрости (Природа/Набор травника) КС 15}, чтобы определить, можете ли вы найти что-то, что можно было бы собрать. Сравните полученный результат с таблицей \textbf{Сбор трав}, чтобы определить максимальное число \textbf{проверок Сбора}, которые вы можете совершить. Каждый отдельный сбор трав занимает \textbf{1 час}; вы не обязаны пользоваться всеми доступными проверками.
	
	Базовый КС составляет 15, однако, это значение может быть скорректировано (или проверка может вообще оказаться недоступной) в зависимости от окружающей среды. Например, если вы путешествуете через скалистые горы, растения могут быть редкими, или же могут вообще не расти на камнях.
	
	Если у вас есть владение набором травника, но нет владения навыком Природа, добавляйте бонус мастерства к проверкам. Если у вас есть владение и тем, и тем, вы получаете \textbf{преимущество} на эту проверку.
	
	\subsubsection*{Направленный поиск}
	
	Вероятно, вы наверняка знаете, что вы ищите. В качестве альтернативы обычной проверке, вы можете выбрать определенный известный ингредиент, который вы хотите найти в текущем биоме. Проведите проверку \textbf{Мудрости (Природа/Набор травника) КС 15 + модификатор КС ингредиента}. В случае провала, вы тратите 1 час и не находите искомый ингридиент. Если вы проводите поиск конкретного ингредиента, то вы можете провести сбор лишь один раз.
	
	\subsubsection*{Далекие путешествия}
	
	В течении светового дня вы можете углубляться в дикую природу, чтобы добираться до удаленных биомов. В зависимости от того, где в мире вы находитесь, вы можете предпринимать путешествия в далекие места при условии что у вас есть достаточно свободного времени чтобы туда дойти, собрать травы и вернуться назад. За каждый день, проведенный в поиске трав, вы можете совершить до двух проверок Травничества.
	
	\section{Сбор растений}
	
	\subsection{Проверка сбора}
	
	Вы успешно нашли растение. Теперь его необходимо собрать. Совершите \textbf{проверки сбора} в количестве вплоть до указанного в таблице \textbf{Сбор трав} значения совершая броски \textbf{1d2, 2d6 и 1d12} для каждого сбора. Каждый отдельный сбор занимает 1 час.
	
	Персонажи, имеющие владение набором травника, имеют большие навыки в сборе ингредиентов и получении бонусов при этом. Они получают +1 бонус к количеству получаемых при сборе ингредиентов при владении набора травника и +2 бонус при наличии \textbf{Компетентности} с ним. Рейнджеры и друиды также получают дополнительный +1 бонус в силу их единства с природой. Эти бонусы суммируются, поэтому рейнджер, имеющий владение набором травника, будет получать +2 ко всем проверкам Сбора при использовании 1d2. Это значит, что они всегда будут находить или извлекать как минимум 3 порции каждого из получаемых ингредиентов.
	
	\paragraph*{Тип ингредиента: 2d6}
	\noindent
	\begin{itemize}
		\item Данный бросок совершается по таблице, соответствующей местности (биому), в которой вы сейчас находитесь. Сравните полученный результат с ней, чтобы определить, какое растение вы обнаружили.
	\end{itemize}
	
	\paragraph*{Эффективность сбора: 1d2}
	
	\noindent\textit{(+1 для Рейнджеров и Друидов)}
	
	\noindent\textit{(+1 за владение набором травника и +2 за компетентность с ним)}
	\noindent
	\begin{itemize}
		\item Результат данного броска определяет, сколько порций обнаруженного растения вы можете получить при сборе. 
	\end{itemize}

	\paragraph*{Тип обычного ингредиента (совершается на всегда): 1d12}
	
	\begin{itemize}
		\item Этот бросок совершается в том случае, если при броске 2d6 вам выпало 6, 7 или 8. В таком случае в таблице текущего биома будет указано <<обычный ингредиент>>. Совершив бросок, обратитесь к таблице <<Обычные ингредиенты>>, чтобы узнать, какое растение вы обнаружили.
	\end{itemize}
	
	\paragraph*{Пример сбора:}
	
	Вы находитесь в лесу и ищете ингредиенты. При совершении бросков вы получили следующий результат:
	
	\begin{itemize}
		\item 1d2 = 2
		\item 2d6 = 4
		\item 1d12 --- не важен при результате на 2d6 меньше 6
	\end{itemize}

	Получается, что согласно таблице \textbf{Лес} вы получили 2 единицы рвоска.
	
	\section{Определение растений}
	
	\subsection{Неизвестные ингредиенты}
	
	Что происходит, если вы находите неизвестный вам до этого момента ингредиент? Проведите проверку \textbf{Интеллекта (Природа/набор травника)}, чтобы понять, можете ли вы определить, что это за растение и каковы его возможные полезные свойства (пассивные значения и предыдущие проверки могут давать автоматический успех). КС этого броска определяется как \textbf{15 + модификатор КС ингредиента}.
	
	Если у вас есть владение набором травника, но нет владения навыком Природа, добавляйте бонус мастерства к проверкам. Если у вас есть владение и тем, и тем, вы получаете \textbf{преимущество} на эту проверку.
	
	В случае успеха вы узнаете название растения и \textbf{формулу} для его использования в зелье или яде.
	
	%Пофиксить страницу
	
	Если у вас не получается определить ингредиент, то вы можете провести с этой целью исследование (XGE 132), или же провести эксперименты. Каждой единицы полученного ингредиента достаточно для проведения одной попытки эксперимента для определения его полезных свойств (см. раздел \textbf{Алхимия} ниже).
	
	\bigskip
	
	\begin{dndtable}
		{\Large \textbf{Ограничение на возможные знания}}
		
		Вероятно, тот, с кем вы общаетесь, обладает ограниченным количеством информации. Это ограничение может проявляться тремя различными способами:
		
		\begin{itemize}
			\item \textbf{Экосистема} (лесные растения, горные, и т.д)
			
			\item \textbf{Редкость} (обычные, необычные, редкие или очень редкие)
			
			\item \textbf{Использование} (зелья или яды)
		\end{itemize}
		
		Алхимик или травник, с которым вы общаетесь, может, к примеру, заниматься только зельями, или же может быть ограничен знанием исключительно тех ингредиентов, которые произрастают неподалеку.
	\end{dndtable}
		
	
	\onecolumn
	\section{Продажа трав}
	
	Травы и обычные растения зачастую продаются в больших и не очень городах, а иногда и в деревнях. Это может происходить либо днем, либо только в определенные периоды, однако способ торговли от этого не зависит. Однако, в зависимости от экономики вашего мира, цены и количество продаваемых ингредиентов может разительно отличаться.
	
	Не ожидайте, что вы придете в город Вилсбури, который недавно был ограблен орками, и продадите ваши запасы трав по лучшей цене. Вряд ли у вас вообще это получится. Иногда вам может повезти и вы продадите все ненужные ингредиенты в столице, которая всегда нуждается в свежих травах, иногда же вам придется припрятать ваши запасы на некоторое время.
	
	Если рассматривать обычные ежедневные обстоятельства, то игрок может продать пригоршню или две обычных ингредиентов торговцу в большом или не очень городе. Однако сумма денег, получаемая за эту сделку, может все равно сильно варьироваться. Редкие ингредиенты обычно тяжело продать за полную цену, а вообще найти для них покупателя обычно еще сложнее.
	
	Как и при продаже магических предметов, игроку необходимо провести проверку \textbf{Интеллекта (Анализ) КС 20}, чтобы найти потенциальных покупателей для их товаров. Другой игрок в группе может помочь в этом непростом деле, предлагая услуги сотоварища, даруя при этом первому игроку преимущество на этот бросок.
	
	В случае провала, покупатель не находится до тех пор, пока игрок не совершит длительный отдых и не предпримет еще одну попытку. В случае успеха, игрок  находит покупателя неподалеку. Если это произошло во время простоя, то для этого должно пройти время, определяемое  редкостью ингредиента. Кроме того, она же может повлиять на вероятность того, что цена окажется не максимальной для отдельно взятого ингредиента. Обратитесь к таблице ниже, чтобы определить как цены, предлагаемые покупателем, так и необходимое число дней да его нахождения.
	
	\begin{dndtable}
	\vspace{-4mm}
	\begin{table}[H]
		
		{\Large\textbf{Продажа трав}}
		
		\centering
		
		\medspace
		
		\begin{tabularx}{\linewidth}{c c c C}
			\textbf{Редкость} & \textbf{Цена} & \textbf{Модификатор d100} & \textbf{Дней на поиск покупателя (шанс за день)} \\
			\cellcolor{altertable}Обычный & \cellcolor{altertable}До 15 зм & \cellcolor{altertable}+10 & \cellcolor{altertable}1d4 (25\% за один день) \\
			Необычный & 16--40 зм & +0 & 1d6 (17\% за один день) \\
			\cellcolor{altertable}Редкий & \cellcolor{altertable}41--100 зм & \cellcolor{altertable}-10 & \cellcolor{altertable}1d8 (13\% за один день) \\
			Очень редкий & 100+ зм & -20 & 1d10 (10\% за один день) \\
		\end{tabularx}
	
		\medspace
		
		\flushleft
		
		Модификатор d100 прибавляется к броску по таблице ниже.
	\end{table}


	\vspace{-4mm}
	\begin{table}[H]
		
		{\Large \textbf{Цены трав}}
		
		\centering
		
		\medspace
		
		\begin{tabularx}{\linewidth}{c X}
			\textbf{d100 + модификатор} & \textbf{Ваш покупатель...} \\
			\cellcolor{altertable}20 или меньше & \cellcolor{altertable}Покупатель предлагает десятую часть базовой цены \\
			21--40 & Покупатель предлагает пятую часть базовой цены. Сомнительный покупатель предлагает половину базовой цены \\
			\cellcolor{altertable}41--80 & \cellcolor{altertable}Покупатель предлагает половину базовой цены. Сомнительный предлагает полную цену \\
			81--90 & Покупатель предлагает купить все ваши ингредиенты за один раз за половину цены \\
			\cellcolor{altertable}90 или больше & \cellcolor{altertable}Сомнительный покупатель предлагает купить все ваши ингредиенты за полную цену без лишних вопросов \\
		\end{tabularx}
	
	\medspace
	
	\flushleft
	
	Личность покупателя предстоит определить ДМу. Иногда, особенно при покупке редких и очень редкий вещей, покупатели могут воспользоваться посредниками для сохранения анонимности. Если покупатель сомнительный, то сделка с ним может иметь последствия в будущем.
		
	\end{table}

\end{dndtable}
	
	
	\twocolumn
	\section{Сбор ингредиентов с существ}
	
	Части существ могут быть использованы в качестве алхимичекских материалов, а также для создания доспехов и оружия для приключенцев, некоторые из них могут даже даровать какие-либо особенности. Кто-то может захотеть взять часть поверженного врага в качестве трофея или для украшения доспеха или дома.
	
	Если вы хотите отделать некоторую часть животного или существа, вы должны провести проверку \textbf{Интеллекта (Природа)}, чтобы определить, можете ли вы понять, как это безопасно и эффективно сделать. В случае успеха вы получаете +5 бонус на дальнейшую проверку Сбора.
	
	После этого вы можете совершить проверку \textbf{Мудрости (Выживание)}, чтобы получить желаемую часть существа. В случае успеха, вы ее получате; в случае неудачи интересующая вас часть либо повреждена, либо уничтожена.
	
	ДМ решает, можно ли при этом применить какие-либо навыки и способности, в зависимости от таких факторов как что вы вообще пытаетесь отделать, тип существа и насколько данное существо редкое в вашем мире.
	
	КС обеих проверок рассчитывается следующим образом: \textbf{КС = 15 + 1/2 Опасность существа}. Если опасность существа меньше чем 2, то она не добавляется к КС.
	
	Число проверок, которые вы можете совершить, и время, необходимое для сбора различных частей создания, зависит от его размера и указано в таблице ниже.
	
	\begin{table}[H]
		
		\centering
		
		\begin{tabularx}{\linewidth}{C C C}
			\textbf{Размер существа} & \textbf{Макс. число проверок сбора} & \textbf{Необходимое время}\\
			\cellcolor{altertable}Крохотное & \cellcolor{altertable}1 & \cellcolor{altertable}15 минут\\
			Маленькое & 1 & 30 минут\\
			\cellcolor{altertable}Среднее & \cellcolor{altertable}2 & \cellcolor{altertable}1 час\\
			Большое & 3 & 2 часа\\
			\cellcolor{altertable}Огромное & \cellcolor{altertable}4 & \cellcolor{altertable}3 часа\\
			Громадное & 5 & 4 часа\\
		\end{tabularx}
	\end{table}
	
	Каждая успешная проверка дает вам число единиц материала/ингредиента, зависящее от размера существа и указанное в таблице ниже.
	
	\begin{table}[H]
		
		\centering
		
		\begin{tabularx}{\linewidth}{c c}
			\textbf{Размер существа} & \textbf{Единиц собрано за проверку}\\
			\cellcolor{altertable}Крохотное & \cellcolor{altertable}1/4 единицы\\
			Маленькое & 1/2 единицы\\
			\cellcolor{altertable}Среднее & \cellcolor{altertable}1 единица\\
			Большое & 2 единицы\\
			\cellcolor{altertable}Огромное & \cellcolor{altertable}3 единицы\\
			Громадное & 4 единицы\\
		\end{tabularx}
	\end{table}
	
	При использовании частей существ для создания брони и оружия вам необходимо число единиц материала, зависящее от размера существа, которое будет пользоваться изготавливаемым предметом. Вы можете прочитать больше о создании предметов и материалах в разделе ремесел.
	
	Когда вы используете собранные части в алхимии, вы можете использовать 1 единицу ингредиента для каждой проверки Ахимии или создания зелья или яда. Заметьте, что некоторое яды, извлекаемые из животных, не требуют алхимической обработки, например, яд виверны.
	
	
	
	\begin{dndtable}
		
		{\Large \textbf{Сбор ядов}}
		
		Если вы пытаетесь собрать яд существа, то делать это необходимо крайне аккуратно. Если вы провалите проверку сбора больше, чем на 5 пунктов, то вы подвергнетесь воздействию этого яда.
		
		Если вы владеете инструментами отравителя, то ваша практика и навыки обращения с ними приводят к отсутствию риска вашего отравления. Кроме того, вы можете добавить ваш бонус мастерства к проверке, если вы не владеете в навыке, необходимом для сбора. Если вы вы владеете одновременно и инструментами отравителя, и необходимым навыком, то вы получаете преимущество на этот бросок.
		
		При сборе яда вы получаете одну единицу яда с существа независимо от его размера. Это число не идет в учет суммарных проверок сбора для существа.
		
	\end{dndtable}

	\subsubsection{Примеры собираемых частей}
	
	Приведенная ниже таблица приводит примеры частей существ, которые можно собрать. Некоторые из них могут причинять повреждения, если у вас не получится их собрать. Урон элементом может быть любым типом урона, которые определяется ДМом (например, персонаж может получить урон огнем при проваленной проверке для сбора огненной железы красного дракона). Вы можете определить нанесенный урон используя те же способы, что и при определении урона ловушек, приведенные в главе 5 \textit{Руководства Мастера}.	
	
	\begin{table}[H]
		
		\centering
		
		
		\begin{tabularx}{\linewidth}{>{\hsize=0.55\hsize}C C}
			\textbf{Часть} & \textbf{Возможные последствия} \\
			\cellcolor{altertable}Жало & \cellcolor{altertable}Наносит урон ядом при проваленной проверке \\
			Крылья, перья & --- \\
			\cellcolor{altertable}Плавники & \cellcolor{altertable}--- \\
			Хитин & --- \\
			\cellcolor{altertable}Хвост & \cellcolor{altertable}--- \\
			Клыки, зубы & Наносят колющий урон при проваленной проверке \\
			\cellcolor{altertable}Внутренние органы & \cellcolor{altertable}---  \\
			Рога & Наносят колющий урон при проваленной проверке \\
			\cellcolor{altertable}Эктоплазма & \cellcolor{altertable}Наносит некротической урон при проваленной проверке \\
			Чешуя & --- \\
			\cellcolor{altertable}Элементальная эссенция & \cellcolor{altertable}Наносит урон элементом при проваленной проверке \\
			Когти & Наносят рубящий урона при проваленной проверке \\
			\cellcolor{altertable}Кости & \cellcolor{altertable}--- \\
			Слизь, выделения & Наносят урон элементом или ядом при проваленной проверке \\
			\cellcolor{altertable}Жизненно важный орган & \cellcolor{altertable}Наносит урон элементом при проваленной проверке  \\
			Мех, шкура & --- \\
			\cellcolor{altertable}Кровь & \cellcolor{altertable}--- \\
		\end{tabularx}
	\end{table}
	

	
	\subsubsection{Ценность различных частей}
	
	Ценность собранных вами ресурсов обычно находится в промежутке от 1\% до 50\% количества очков опыта, получаемых за победу над существом. Вы можете определить ценность каждой из частей с помощью таблицы ниже.
	
	\textbf{Например}, если вы собираете несколько перьев гиппогрифа (класс опасности 1), ценность каждого из них будет составлять 1\% от базового опыта, получаемого за него (200 очков опыта), что составляет 2 зм. Единица чешуи псевдодракона будет стоить 5 см (класс опасности 1/4), а за единицу чешуи взрослого синего дракона можно выручить до 1500 зм (класс опасности 16).
	
	\begin{center}
	
	\begin{table}[H]
		
		{\Large\textbf{Цена за единицу}}
		
		\centering
		
		\medspace
		
		\begin{tabularx}{\linewidth}{C C C}
			\textbf{ПО существа} & \textbf{Редкость существа} & \textbf{Цена за единицу (в зм)} \\
			\cellcolor{altertable}4 или меньше & \cellcolor{altertable}Обычное & \cellcolor{altertable}1\% опыта за существо \\
			5--12 & Необычное & 5\% опыта за существо \\
			\cellcolor{altertable}13--18 & \cellcolor{altertable}Редкое & \cellcolor{altertable}10\% опыта за существо \\
			19--24 & Очень редкое & 25\% опыта за существо  \\
			\cellcolor{altertable}$25+$ & \cellcolor{altertable}Легендарное & \cellcolor{altertable}50\% опыта за существо \\
		\end{tabularx}
\end{table}
	\end{center}
	
	\subsubsection{Свежевание (вариант для заготовки провизии)}
	
	Помимо того, что персонажи могут заготавливать припасы для выживания в дикой природе, они также могут охотиться на существ, чтобы получить мясо. Полученное таким образом мясо испортится через день, если его не съесть и не сохранить каким-либо образом. Поедание испорченного мяса требует спасброска \textbf{Телосложения} для избежания тошноты или какого либо заболевания.
	
	Персонаж может совершить проверку \textbf{Мудрости (Выживание) КС 15}, чтобы предпринять попытку освежевать тушу. КС 15 типичен, однако может быть скорректирован в зависимости от ситуации. Количество получаемого мяса и необходимое для свежевания время определяется с размером существа с помощью таблицы ниже.
	
	Сбор мяса не идет в учет суммарных проверок сбора для существа.
	
		\begin{table}[H]
			
			\centering
			
			\begin{tabularx}{\linewidth}{C C C}
				\textbf{Размер существа} & \textbf{Максимальная масса мяса} & \textbf{Необходимое время} \\
				\cellcolor{altertable}Крохотное & \cellcolor{altertable}1 фунт & \cellcolor{altertable}15 минут \\
				Маленькое & 4 фунта & 30 минут \\
				\cellcolor{altertable}Среднее & \cellcolor{altertable}16 фунтов & \cellcolor{altertable}1 час \\
				Большое & 32 фунта & 2 часа \\
				\cellcolor{altertable}Огромное & \cellcolor{altertable}64 фунта & \cellcolor{altertable}3 часа \\
			\end{tabularx}
		\end{table}
	
	
	
	\section{Harvesting the Earth}
	
	Такие материалы, как древесина, камень, минералы и кораллы могут быть использованы для создания оружия и доспехов. Каждая попытка поиска и потенциального сбора подобных материалов требует \textbf{1 дня}.
	
	В качестве способа провести время простоя, вы можете предпринять путешествие в удаленные места, при условии, что у вас достаточно времени, чтобы туда дойти, собрать ресурсы, и вернуться назад. В зависимости от того, сколько материалов вы соберете, вам может понадобиться позаботиться о транспорте.
	
	Минералы делятся на два типа, \textbf{руды} и \textbf{драгоценные камни}.
	
	\subsubsection{Поиск и определение ресурсов}
	
	Для того, чтобы определить месторождение минералов или других материалов, совершите проверку \textbf{Мудрости (Природа/Набор инструментов)}. Сравните результат с таблицей ниже. Вы находите материалы, редкость которых равна или меньше полученного результата. ДМ определяет, какие материалы доступны там, где вы в данный момент находитесь.
	
	Вы совершаете проверку \textbf{Интеллекта (Природа/Набор инструментов)} против КС, указанного в той же таблице, для определения этих материалов (пассивные значения и предыдущие проверки могут давать автоматический успех). В случае успеха, вы знаете, какой материал вы получили, и каковы его полезные свойства. В случае провала, материал вам неизвестен. Вы можете пытаться определить его с помощью экспериментов, или же просто сделав из него что-то и посмотрев на полученный результат.
	
	Для каждой из этих проверок, если вы владеете подходящим инструментом, но при этом не владеет навыком Природа, добавьте свой бонус мастерства. Если же вы владеете и тем, и тем. то вы получаете преимущество на эту проверку.
	
	\begin{table}[H]
		
		\centering
		
		\begin{tabularx}{0.8\linewidth}{c C}
			\textbf{КС} & \textbf{Редкость материала} \\
			\cellcolor{altertable}15 & \cellcolor{altertable}Обычный \\
			17 & Необычный \\
			\cellcolor{altertable}19 & \cellcolor{altertable}Редкий \\
			21 & Очень редкий \\
		\end{tabularx}
	\end{table}
	
	
	Для нахождения какого-то конкретного материала вам необходимо успешно пройти проверку \textbf{Мудрости (Природа/Набор инструментов)} против КС, соответствующего его редкости. В случае провала, вы не находите \textbf{ничего} полезного
	
	\subsubsection{Сбор ресурсов}
	
	Из всех найденных материалов вам необходимо выбрать один, который вы будете собирать. Базовый КС для проверки Сбора составляет \textbf{15} при использовании подходящего инструмента (например, топора для рубки дерева, кирки для руд и т.д.). Этот КС может быть скорректирована в зависимости от обстоятельств.
	
	Совершите проверку навыка (или владения набором инструментов), подходящего для материала, который вы собираетесь собрать, например, \textbf{Силы (Атлетика/Набор инструментов)} для горного дела или рубки дерева, или же \textbf{Ловкости (набор инструментов)} для извлечения ценных камней. В случае успеха, вы получаете число единиц, равное \textbf{2d4 + модификатор Телосложения}. В случае провала, вы получаете половину от названного числа.
	
	При сборе драгоценных камней, ДМ может либо сам определить, сколько вы их нашли, либо бросить d20 и сравнить результат с таблицей. Вы можете определить, какие конкретно камни были найдены, с помощью таблиц в руководстве мастера (\textit{DMG} 134)
	
	\begin{table}[H]
		
		\centering
		
		\begin{tabularx}{\linewidth}{c C}
			
			\textbf{d20} & \textbf{Число найденных самоцветов} \\
			
			\cellcolor{altertable}$1-10$ & \cellcolor{altertable}1 драгоценный камень (10 зм) \\
			
			$11-15$ & 1d4+1 драгоценный камень (10 зм) \\
			
			\cellcolor{altertable}$16-18$ & \cellcolor{altertable}1 драгоценный камень (50 зм) \\
			
			19 & 1d4+1 драгоценный камень (50 зм) \\
			
			\cellcolor{altertable}20 & \cellcolor{altertable}Бросьте дважды \\
			
		\end{tabularx}
	\end{table}
	
	
	
	\subsubsection{Ценность полученных материалов}
	
	Минералы и все прочее, что вы собираете, может быть продано, и, чаще всего, куплено. Цена за единицу будет зависеть от того, что именно вы покупаете. Для большей информации смотрите \textbf{materials field guide}.
	
	\subsubsection{Особенные материалы для биома}
	
	Таблицы ниже могут быть использованы ДМом для определения, что доступно для сбора в каждой из приведенных локаций. Не везде в биоме обязательно встречаются все приведенные ресурсы, кроме того, одни из них могут встречаться реже. 
	
	
	
	\begin{table}[H]

		{\Large \textbf{Равнина}}
		
		\centering
		
		\medspace 
		
		\begin{tabularx}{\linewidth}{C C}
			
			\textbf{Редкость} & \textbf{Ресурс} \\
			
			\cellcolor{altertable}Обычный & \cellcolor{altertable}Темная сталь \\
			
			Обычный & Темная древесина \\
			
			\cellcolor{altertable}Обычный & \cellcolor{altertable}Плетеные листья \\
			
			Необычный & Пятнистая береза \\
			
			\cellcolor{altertable}Редкий & \cellcolor{altertable}Дуб Асмороха \\
			
		\end{tabularx}
	\end{table}

	\begin{table}[H]

		{\Large \textbf{Холмы}}
		
		\centering
		
		\medspace 
		
		\begin{tabularx}{\linewidth}{C C}
			
			\textbf{Редкость} & \textbf{Ресурс} \\
			
			\cellcolor{altertable}Обычный & \cellcolor{altertable}Холодное железо \\
			
			Обычный & Темная сталь \\
			
			\cellcolor{altertable}Необычный & \cellcolor{altertable}Адаманит \\
			
			Необычный & Мифрил \\
			
			\cellcolor{altertable}Редкий & \cellcolor{altertable}Дварфийский камень \\
			
		\end{tabularx}
	\end{table}

	\begin{table}[H]

		{\Large \textbf{Лес}}
		
		\centering
		
		\medspace 
		
		\begin{tabularx}{\linewidth}{C C}
			
			\textbf{Редкость} & \textbf{Ресурс} \\
			
			\cellcolor{altertable}Обычный & \cellcolor{altertable}Темная древесина \\
			
			Обычный & Плетеные листья \\
			
			\cellcolor{altertable}Необычный & \cellcolor{altertable}Ядовитый Коа \\
			
			Необычный & Пятнистая береза \\
			
			\cellcolor{altertable}Редкий & \cellcolor{altertable}Дуб Асмороха \\
			
		\end{tabularx}
	\end{table}

	\begin{table}[H]

		{\Large \textbf{Горы}}
		
		\centering
		
		\medspace 
		
		\begin{tabularx}{\linewidth}{C C}
			
			\textbf{Редкость} & \textbf{Ресурс} \\
			
			\cellcolor{altertable}Обычный & \cellcolor{altertable}Холодное железо \\
			
			Обычный & Темная сталь \\
			
			\cellcolor{altertable}Необычный & \cellcolor{altertable}Адаманит \\
			
			Необычный & Мифрил \\
			
			\cellcolor{altertable}Необычный & \cellcolor{altertable}Обсидиан \\
			
			Редкий & Воздушный кристалл \\
			
			\cellcolor{altertable}Редкий & \cellcolor{altertable}Дварфийский камень \\
			
			Редкий & Вечный лед \\
			
			\cellcolor{altertable}Редкий & \cellcolor{altertable}Игнум \\
			
			Редкий & Орихалк \\
			
			\cellcolor{altertable}Очень редкий & \cellcolor{altertable}Небесное железо \\
			
		\end{tabularx}
	\end{table}

	\begin{table}[H]

		{\Large \textbf{Пустыня}}
		
		\centering
		
		\medspace 
		
		\begin{tabularx}{\linewidth}{C C}
			
			\textbf{Редкость} & \textbf{Ресурс} \\
			
			\cellcolor{altertable}Различная & \cellcolor{altertable}Эллондийская шкура \\
			
			Редкий & Игнум \\
			
			\cellcolor{altertable}Редкий & \cellcolor{altertable}Орихалк  \\
			
		\end{tabularx}
	\end{table}
	
	\begin{table}[H]

		{\Large \textbf{Болото}}
		
		\centering
		
		\medspace 
		
		\begin{tabularx}{\linewidth}{C C}
			
			\textbf{Редкость} & \textbf{Ресурс} \\
			
			\cellcolor{altertable}Обычный & \cellcolor{altertable}Темная древесина \\
			
			Обычный & Плетеные листья \\
			
			\cellcolor{altertable}Необычный & \cellcolor{altertable}Ядовитый Коа \\
			
			Необычный & Пятнистая береза \\
			
			\cellcolor{altertable}Редкий & \cellcolor{altertable}Дуб Асмороха \\
			
		\end{tabularx}
	\end{table}
		
	\begin{table}[H]

		{\Large \textbf{Подземье}}
		
		\centering
		
		\medspace 
		
		\begin{tabularx}{\linewidth}{C C}
			
			\textbf{Редкость} & \textbf{Ресурс} \\
			
			\cellcolor{altertable}Обычный & \cellcolor{altertable}Холодное железо \\
			
			Необычный & Адаманит \\
			
			\cellcolor{altertable}Необычный & \cellcolor{altertable}Мифрил \\
			
			Необычный & Обсидиан \\
			
			\cellcolor{altertable}Редкий & \cellcolor{altertable}Воздушный кристалл \\
			
			Редкий & Дварфийский камень \\
			
			\cellcolor{altertable}Редкий & \cellcolor{altertable}Игнум \\
			
			Редкий & Орихалк \\
			
			\cellcolor{altertable}Очень редкий & \cellcolor{altertable}Небесное железо \\
			
		\end{tabularx}
	\end{table}

	\begin{table}[H]

		{\Large \textbf{Арктика}}
		
		\centering
		
		\medspace
		
		\begin{tabularx}{\linewidth}{C C}
			
			\textbf{Редкость} & \textbf{Ресурс} \\
			
			\cellcolor{altertable}Обычный & \cellcolor{altertable}Холодное железо \\
			
			Редкий & Вечный лед \\
			
		\end{tabularx}
	\end{table}
		
	\begin{table}[H]

		{\Large \textbf{Побережье}}
		
		\centering
		
		\medspace
		
		\begin{tabularx}{\linewidth}{C C}
			
			\textbf{Редкость} & \textbf{Ресурс} \\
			
			\cellcolor{altertable}Обычный & \cellcolor{altertable}Коралл \\
			
			Необычный & Обсидиан \\
			
			\cellcolor{altertable}Необычный & \cellcolor{altertable}Ядовитый Коа \\
			
			Редкий & Воздушный кристалл \\
			
		\end{tabularx}
	\end{table}
		
		
	
	\begin{dndtable}
			
			{\Large\textbf{Прочие ресурсы}}
			
			Возможно, игроки будут искать что-то определенное, не указанное здесь. В таком случае, приведенные материалы можно использовать в качестве отправной точки для создания нового материала. Он должен как минимум иметь редкость, цену за единицу и быть полезным либо для создания оружия или брони (или применяться где-то еще), даже если эффект чисто декоративным.
	\end{dndtable}
	
	
	
	\onecolumn
	
	\section{Общий перечень материалов}
		
		\begin{table}[H]
			
			\centering 
			
			\begin{tabularx}{0.8\linewidth}{C c C}
				
				\textbf{Ресурс} & \textbf{Редкость} & \textbf{Источник/Биом} \\
				
				\cellcolor{altertable}Адаманит & \cellcolor{altertable}Необычный & \cellcolor{altertable}Горы, Холмы, Подземье \\
				
				Воздушный кристалл & Редкий & Горы, Подземье, Побережье \\
				
				\cellcolor{altertable}Дуб Асмороха & \cellcolor{altertable}Редкий & \cellcolor{altertable}Лес, Болото, Равнина \\
				
				Перья чудовища & Различная & Любое пернатое существо \\
				
				\cellcolor{altertable}Кости & \cellcolor{altertable}Различная & \cellcolor{altertable}Любое существо со скелетом \\
				
				Хитин & Различная & Любое существо с панцирем \\
				
				\cellcolor{altertable}Холодное железо & \cellcolor{altertable}Обычный & \cellcolor{altertable}Горы, Холмы, Арктика, Подземье \\
				
				Кораллы & Обычный & Побережье \\
				
				\cellcolor{altertable}Темная сталь & \cellcolor{altertable}Необычный & \cellcolor{altertable}Холмы, Равнина, Горы \\
				
				Темная древесина & Обычный & Лес, Равнина, Болото \\
				
				\cellcolor{altertable}Дварфийский камень & \cellcolor{altertable}Редкий & \cellcolor{altertable}Горы, Холмы, Подземье \\
				
				Эллондийская шкура & Различная & Пустынные существа  \\
				
				\cellcolor{altertable}Вечный лед & \cellcolor{altertable}Редкий & \cellcolor{altertable}Арктика, Горы (вершины) \\
				
				Игнум & Редкий & Пустыня, Подземье, Горы (вулканы) \\
				
				\cellcolor{altertable}Инфернальная кожа & \cellcolor{altertable}Очень редкий & \cellcolor{altertable}Инфернальные планы \\
				
				Инфернальная сталь & Очень редкий & Инфернальные планы \\
				
				\cellcolor{altertable}Плетеные листья & \cellcolor{altertable}Обычный & \cellcolor{altertable}Лес, Равнина, Болото \\
				
				Мифрил & Необычный & Горы, Холмы, Подземье \\
				
				\cellcolor{altertable}Чешуя монстра & \cellcolor{altertable}Различная & \cellcolor{altertable}Любое существо с чешуей \\
				
				Обсидиан & Необычный & Горы, Побережье, Подземье (вулканы) \\
				
				\cellcolor{altertable}Орихалк & \cellcolor{altertable}Редкий & \cellcolor{altertable}Пустыня, Горы, Подземье \\
				
				Ядовитый Коа & Необычный & Лес, Болото, Побережье \\
				
				\cellcolor{altertable}Теневой шелк & \cellcolor{altertable}Различная & \cellcolor{altertable}Гнезда паукообразных существ \\
				
				Шаед & Очень редкий & Царство Теней \\
				
				\cellcolor{altertable}Пятнистая береза & \cellcolor{altertable}Необычный & \cellcolor{altertable}Лес, Равнина, Болото \\
				
				Небесное железо & Очень редкий & Подземье, Горы \\
				
			\end{tabularx}
		\end{table}
	
	\twocolumn
	\chapter{Алхимия}
	
	Хотя искусный травник может сделать целительную мазь с помощью корня Дикого Шалфея, она сможет залечить раны путешественников только в долгосрочной перспективе (обычно за несколько дней или недель) и только до определенной степени, зависящей от тяжести повреждений. Для большинства простолюдинов это единственное лечение, которое они могут найти или позволить себе.
	
	Алхимия, напротив, является процессом попыток выделить, и усовершенствовать ингредиенты (как растительные, так и добытые с существ), развить их свойства. Приложенное ниже руководство обзорно осмотрит, как с помощью алхимии можно экспериментировать с природными материалами для того, чтобы раскрыть их потенциал и открыть \textbf{магические формулы}.
	
	Когда у игроков есть формула и ингредиенты, они могут затратить необходимые время и деньги для создания предмета (XGE 128).
	
	Для проведения алхимического эксперимента, у вас должен быть \textbf{набор алхимика}, а также вам необходимо пройти проверку \textbf{Алхимии}.
	
	\subsection{Яды}
	
	Одни из самых типичный инструментов убийц. Коварная комбинация красоты и смертельной опасности.
	
	Каждый яд, чтобы произвести свой эффект, должен быть приведен в контакт с жертвой одним из следующих способов:
	
	\begin{itemize}
		\item Через дыхательные пути
		
		\item Через пищеварение
		
		\item Через рану
		
		\item Через кожу
	\end{itemize}
	
	\begin{dndtable}
		
		{\Large \textbf{Отравление оружия}}
		
		Один флакон яда можно нанести на 1 оружие или на 3 боеприпаса. Использованный таким образом яд сохраняет свои свойства до тех пор, пока оружие не совершит успешной атаки по существу или пока его не смоет.
		
	\end{dndtable}
	
	
	
	\subsection{Проверка Алхимии}
	
	\textbf{Интеллект (Магия/набор Алхимика)}
	
	\textbf{КС = 15 + модификатор сложности ингредиента}
	
	Каждая попытка требует \textbf{1 сутки} и 1 единицу тестируемого материала. При проведении проверки Алхимии возможно три возможных исхода:
	
	\begin{itemize}
		\item В случае успешной проверки, вы создаете соответствующее зелье или яд и изучаете его формулу.
		
		\item В случае провала содержимое флакона выглядит совершенно не так, как вы ожидали. ДМу решать, узнает ли персонаж, что у него не получилось, или нет.
		
		\item Если проверка провалена меньше, чем на 3, то вы совершаете некоторый прогресс. Зелье получается не полностью, однако для последующих попыток КС понижается на 2.
	\end{itemize}

	\subsubsection{Обратное конструирование}
	
	Если у вас есть готовое зелье или яд, то вы можете попытаться применить метод обратного конструирования, чтобы узнать его формулу. Каждый флакон зелья или яда предоставляет три единицы для ваших экспериментов. Проведите проверку \textbf{Алхимии} против КС, зависящего от редкости образца.
	
	\begin{table}[H]
		
		\centering
		
		\begin{tabularx}{0.8\linewidth}{c C}
			
			\textbf{КС} & \textbf{Редкость смеси} \\
			
			\cellcolor{altertable}10 & \cellcolor{altertable}Обычный \\
			
			15 & Необычный \\
			
			\cellcolor{altertable}20 & \cellcolor{altertable}Редкий \\
			
			25 & Очень редкий \\
			
			\cellcolor{altertable}30 & \cellcolor{altertable}Легендарный \\
			
		\end{tabularx}
	
	\medspace
	
	\flushleft
	
	Все, что останется после экспериментов, будет иметь ослабленный эффект.
	\end{table}
	
	
	

	
	\subsubsection{Создание ядов и зелий}
	
	Если вы знаете формулу и имеете доступ к необходимым ингредиентам, инструментам (инструменты алхимика для зелий и отравителя для ядов), времени и деньгам, вы можете создать зелья и яды, потратив по 1 единице каждого ингредиента, который вы бы хотели использовать и воспользовавшись таблицей ниже.
	
	Отметим, что время для создания зелий лечения отличается и составляет 1/5/15/20 дней в зависимости от редкости.
	
	\begin{table}[H]
		
		\centering
		
		\begin{tabularx}{\linewidth}{C C C}
			
			\textbf{Редкость ингредиентов} & \textbf{Дней (Зелье/Яд)} & \textbf{Стоимость} \\
			
			\cellcolor{altertable}Обычный & \cellcolor{altertable}$2/1$ & \cellcolor{altertable}25 зм \\
			
			Необычный & $5/2$ & 100 зм \\
			
			\cellcolor{altertable}Редкий & \cellcolor{altertable}$25/12$ & \cellcolor{altertable}1000 зм \\
			
			Очень редкий & $60/30$ & 10000 зм \\
			
		\end{tabularx}
		
	\end{table}
	
	
	
	Смотри таблицы Формул в Аппендиксе А для дополнительной информации.
	
	\chapter{Создание предметов и материалы}
	
	Во время приключений и путешествий вы можете собрать самые различные материалы, необходимые для создания особенной экипировки. Этот раздел расширяет систему крафтов, приведенную в \textit{Руководстве игрока} и модифицирует некоторые ее части.
	
	Вы можете изготавливать немагические предметы, включая снаряжение для путешествий. Для этого не требуется владением каким-то набором инструментов, хотя это, разумеется, помогло бы делу. Кроме того, вам может потребоваться доступ к определенному месту для изготовления предмета. Например, изготовление доспеха или меча по чертежу требует наличия кузницы.
	
	За каждый день простоя, который вы тратите на изготовление предмета, вы можете изготовить один или несколько предметов, суммарная рыночная цена которых не превосходит 25 зм. Кроме того, вы должны затратить сырье, ценой не меньше, чем половина рыночной цены изделия. Если вы хотите изготовить что-то дороже 25 зм, то вы каждый день совершаете прогресс в размере 25 зм, пока не достигните рыночной цены предмета. Например, комплект лат (рыночная цена 1500 зм) потребует для создания 60 дней. К конце процесса суммарная его цена для вас составит половину рыночной цены (эти деньги пойдут на покупку сырья).
	
	Если вы \textbf{не владеете} требуемым набором инструментов, время и стоимость изготовления увеличивается на 50\%. Число дней, которое нетренированный проводит за изготовлением изделия, считается в общее число дней, необходимое для получения владения набором инструментов.
	
	Несколько персонажей могут объединить свои усилия по созданию предмета, при условии, что у всех персонажей есть владение необходимыми наборами инструментов и работают они в одном месте. Каждый персонаж вносить по 25 зм за каждый день работы над предметом. Например, если три персонажа с владениями необходимыми наборами инструментов работают над латами, то они могут закончить работу за 20 дней.
	
	При изготовлении предметов вы можете поддерживать скромное существование, не затрачивая деньги, или же вести комфортный образ жизни за половину цены (смотри главу 5 \textit{Руководства игрока} для большей информации о затратах на существование).
	
	\subsection{Типы предметов}
	
	Существует три основных типа немагических предметов, которые вы можете изготовить:
	
	\begin{itemize}
		\item \textbf{Обычные предметы} это те предметы, которые вы можете с легкостью найти в магазинах, подземельях, а большинство ремесленников делают их практически каждый день. Это может быть, например, длинный меч или кожаный доспех.
		
		\item \textbf{Особенные предметы} это элементы снаряжения, изготавливаемые из особых материалов, например, длинный меч из темной стали.
		
		\item \textbf{Уникальные предметы} это совершенно новые творения с уникальными формами, качествами и бонусами, например, двуручный меч с механизмом, превращающим его в два скимитара.
	\end{itemize}

	\subsubsection{Стоимость особенных предметов}
	
	Рыночная стоимость особенных предметов считается как рыночная стоимость созданного предмета плюс стоимость  каждой единицы особенного материала, использованного при создании.
	
	Например, стоимость длинного меча из темной стали, окантованного обсидианом, рассчитывается следующим образом: цена \textbf{длинного меча} (15 зм) + 2 единицы \textbf{темной стали} (2 x 250 зм) + 2 единицы \textbf{обсидиана} (2 x 250 зм) = 1015 зм.
	
	\subsubsection{Найм ремесленников}
	
	Вы можете нанять одного или нескольких ремесленников, чтобы те помогли вам создать предмет или же чтобы они создали предмет от и до. При использовании особенных материалов, вам необходимо нанять тех, кто умеет с ними работать.
	
	Цена найма ремесленника зависит от того, предмет какого типа вы хотите сделать. В общем случае, для более редких типов предметов вам необходимы те, кто могут работать с материалами и имеют чертеж (или что-то подобное). Вы можете посмотреть стоимость найма работников в таблице ниже. Если вы нанимаете работника, чтобы создать уникальный предмет из особенных материалов, стоимость работ за день составит 15 зм.
	
	\begin{table}[H]
		
		{\Large \textbf{Найм ремесленника}}
		
		\medspace
		
		\centering
		
		\begin{tabularx}{\linewidth}{C C}
			
			\textbf{Тип предмета} & \textbf{Стоимость за день} \\
			
			\cellcolor{altertable}Обычный (обычная работа ремесленника) & \cellcolor{altertable}2 зм \\
			
			Особенный (особые материалы) & 5 зм \\
			
			\cellcolor{altertable}Уникальный (совершенно новый предмет) & \cellcolor{altertable}$10+$ зм ($15+$ зм с особыми материалами) \\
			
		\end{tabularx}
	\end{table}
	
	
	\subsubsection{Создание уникального предмета}
	
	Создание уникального предмета по своей сущности является экспериментом и несет неизбежный риск провала. При создании такого предмета вы должны проводить проверку навыка в конце процесса, чтобы определить, получилось ли у вас. Проверка навыка должна быть подходящей для задачи и используемого материала. Например, Интеллект (Атлетика/Набор инструментов) для кузнечного дела или Ловкость (Природа/Набор инструментов), если вы вырезаете по дереву или работаете с кожей.
	
	Если вы владеете используемым набором инструментов, но не владеете необходимым навыком, вы прибавляете к проверке бонус мастерства. Если же вы владеете и тем, и тем, вы получаете преимущество на эту проверку.
	
	\textbf{КС = 13 + модификатор редкости материала + число различных \textit{особых} материалов + число используемых компонентов}.
	
	Если вы используете больше одного особого материала, используйте для определения КС редкость самого редкого из них.
	
	В случае успеха, наши вам поздравления! Вы успешно создали новый, уникальный предмет. В случае провала, вы должны завершить длительный отдых, прежде чем вы можете провести проверку заново. После трех проваленных проверок, проект признается невозможным для дальнейших попыток его починить, материалы оказываются утерянными, и приходится начинать сначала.
	
	\begin{table}[H]
		
		{\Large \textbf{Модификатор редкости материалов}}
		
		\medspace 
		
		\centering 
		
		\begin{tabularx}{\linewidth}{C C}
			
			\textbf{Редкость} & \textbf{Модификатор КС} \\
			
			\cellcolor{altertable}Не особенный & \cellcolor{altertable}$+0$ \\
			
			Обычный & $+1$ \\
			
			\cellcolor{altertable}Необычный & \cellcolor{altertable}$+2$ \\
			
			Редкий & $+3$ \\
			
			\cellcolor{altertable}Очень редкий & \cellcolor{altertable}$+4$ \\
			
		\end{tabularx}
	\end{table}
	
	\begin{dndtable}
		
		{\Large \textbf{Требования особых уникальных предметов}}
		Отметим, что для каждого компонент, используемого в предмете, требуется полный набор материалов. К примеру, для изготовления двойного кинжала из темной стали потребуется 4 единицы темной стали, по две на каждый кинжал. КС в таком случае составит 13 + 2 (модификатор от темной стали) + 1 (один тип особого материала) + 2 (в изделии используется \textit{два} кинжала) = 18.
		
		Еще пример: для изготовления нагрудника из холодного железа, пронизанного вечным льдом, потребует 3 единицы холодного железа и 3 единицы вечного льда. КС составит 13 + 3 (модификатор вечного льда) + 2 (два различных особых материала) + 1 (\textit{один} нагрудник) = 19.
		
	\end{dndtable}
	
	
	
	\subsection{Особые материалы}
	
	Для создания предмета из особых материалов вам необходимо число единиц материала в соответствии с тем, что вы создаете, и для какого существа. Для существа среднего размера вам необходимо \textbf{3 единицы} одного материала для создания брони и одежды, \textbf{2 единицы} для оружия и щитов и \textbf{1 единицу} для 10 боеприпасов. Время и стоимость изготовления предмета из особых материалов рассчитывается по рыночной цене типа изготавливаемого предмета.
	
	Например, если вы хотите создать доспех из адаманитовых пластин для существа среднего размера, это займет 60 дней, 750 зм и 3 единицы адаманита. Если вы наймете двух ремесленников себе в помощь, то это займет 20 дней, 950 зм и 3 единицы адаманита.
	
	За каждый размер существа больше среднего вам необходимо в два раза больше материалов (на 1 комплект доспехов уйдет 6 единиц, если размер \textbf{большой}, 12 единиц, если \textbf{огромный} и 24, если \textbf{колоссальный}). Аналогично, за каждый размер меньше среднего, количество необходимых материалов \textbf{уменьшается в два раза}.
	
	Создание предмета из двух и более особенных предметов это более деликатный и продвинутый процесс. Такой предмет считается \textbf{Уникальным} при расчете затрат на его изготовления.
	
	В случае использования частей существ, вы можете получить дополнительный бонус, зависящий от уровня опасности существа, согласно таблице ниже.
	
	Например, чешуйчатый доспех из чешуи взрослого синего дракона (уровень опасности 16) будет давать вам дополнительный +1 к классу доспеха, что в сумме будет давать КД = 15 + модификатор ловкости (максимум +2).
	
	\begin{table}[H]
		\begin{tabularx}{\linewidth}{C C C}
			
			\textbf{ПО существа} & \textbf{Бонус к броне} & \textbf{Бонус к оружию} \\
			
			\cellcolor{altertable}4 или меньше & \cellcolor{altertable}$+0$ & \cellcolor{altertable}$+0$ \\
			
			$5 - 12$ & $+0$ & $+1$ \\
			
			\cellcolor{altertable}$13 - 18$ & \cellcolor{altertable}$+1$ & \cellcolor{altertable}$+2$ \\
			
			$19 - 24$ & $+2$ & $+3$ \\
			
			\cellcolor{altertable}$25 + $ & \cellcolor{altertable}$+3$ & \cellcolor{altertable}$+4$ \\
			
		\end{tabularx}
	\end{table}
	
	
	\begin{dndtable}
		
		{\Large \textbf{Жизнь в обучении}}
		
		Тот факт, что вы не знаете, как использовать молот и наковальню, не значит, что вы не можете этому обучиться!
		
		Число дней. которое потребуется для получения владения новым инструментом или языком любого типа составит \boxed{\textbf{50 - 5 x модификатор Интеллекта}}. Отрицательные модификаторы не могут повысить время обучения.
		
		ДМ может решить, что навыки, которыми вы уже владеете, могут внести свой вклад в обучение чему-то новому. Например, высокая Сила может уменьшить время, требуемое для обучения кузнечному делу, а Ловкость может помочь в изучении искусства подделывания предметов или в ювелирном деле. И наоборот, обучение чему-то может потребовать определенного минимального значения какого-то параметра. Вряд ли волшебник с Силой 8 когда-то сможет стать мастером-кузнецом.
		
		Стоимость обучения новому навыку составляет 25 зм за рабочую неделю (5 дней). Она может варьироваться по целому ряду причин, например, ваш инструктор может давать вам скидку за вашу дружбу.
		
	\end{dndtable}
	
	
	
	\titleformat{\chapter}
	{}
	{}
	{8pt}
	{\fontsize{1.113cm}{5}\bfseries\color{sectioncolor}\filright}
	
	\chapter{Полевое руководство по материалам}
	
	Здесь вы найдете список особых материалов, их стоимости за единицу и свойства, которые они даруют доспехам и оружию. Некоторые из них могут иметь какие-то еще свойства, определяемые вашим Мастером (например, доспехи из кости дракона могут даровать вам сопротивление типу урона, причиняемого драконом).
	
	\partc[Необычный материал]{Адаманит}
	
	\noindent Блестяще-черный металл, который называют одним из самый прочных известных материалов.
	
	\begin{itemize}
		\item \textbf{Стоимость за единицу:} 250 зм
		
		\item \textbf{Доспех:} любое критическое попадание по вам становится обычным попаданием.
		
		\item \textbf{Оружие:} атаки игнорируют сопротивление любым немагическим сопротивлениям (например, сопротивление дробящему или колющему урону).
	\end{itemize}

	\partc[Редкий материал]{Воздушный кристалл}
	
	\noindent Светло-голубой кристалл, похожий на стеклянный, для создания легкого, изящного оружия.
	
	\smallskip
	
	\begin{itemize}
		\item \textbf{Цена за единицу:} 500 зм
		
		\item \textbf{Оружие:} получает свойство \textit{легкое}.
	\end{itemize}

	\partc[Редкий материал]{Дуб Асмороха}
	
	\noindent Это твердая древесина была опалена дочерна. На ощупь она очень холодная, создается впечатление, что она совершенно лишена жизни.
	
	\begin{itemize}
		\item \textbf{Цена за единицу:} 500 зм
		
		\item \textbf{Доспех:} некротический урон, причиняемый вам, уменьшается на 1d10.
		
		\item \textbf{Оружие:} когда вы наносите удар по существу, вы наносите дополнительные 1d4 некротического урона.
	\end{itemize}
	
	\partc[Редкость варьируется]{Перья монстров}
	
	\noindent У гиппогрифов, рух, огромных орлов и прочих летающих существ невероятно красивые перья, подчеркивающие их величие.
	
	\begin{itemize}
		\item \textbf{Цена за единицу:} определяется уровнем опасности существа.
		
		\item \textbf{Доспех:} при ношении одежды или неметаллического доспеха, сделанного с использованием этих перьев, вы получаете преимущество на проверки \textbf{Ухода за  животных}
	\end{itemize}

	\partc[Редкость варьируется]{Кости}
	
	\noindent Несколько костей, соединенных вместе, зачастую используются шаманами и дикарями. Они делают вашу внешность более свирепой.
	
	\begin{itemize}
		\item \textbf{Цена за единицу:} определяется уровнем опасности существа.
		
		\item \textbf{Доспех:} вы получаете преимущество на проверки \textbf{Запугивания}. Возможны дополнительные эффекты, определяемые ДМом.
		
		\item \textbf{Оружие:} вы получаете преимущество на проверки \textbf{Запугивания}. Возможны дополнительные эффекты, определяемые ДМом.
	\end{itemize}
	
	\partc[Редкость варьируется]{Хитин}
	
	\noindent Легкие, гибкие панцири различных существ, например гигантских крабов или реморазов, могут применяться в создании доспехов.
	
	\begin{itemize}
		\item \textbf{Цена за единицу:} определяется уровнем опасности существа.
		
		\item \textbf{Доспех:} максимальных бонус ловкости для средних доспехов увеличивается на 1. Требования к Силе у тяжелых доспехов снижаются на 1.
	\end{itemize}

	\partc[Обычный материал]{Холодное железо}
	
	\noindent Это железо как будто бы излучает неестественных холод. Известно своей эффективностью против фей.
	
	\begin{itemize}
		\item \textbf{Цена за единицу:} 100 зм
		
		\item \textbf{Доспех:} Любое существо-фея, вступающее в контакт с доспехом, получает 1d4 урона холодом.
		
		\item \textbf{Оружие:} При попадании по фее вы можете бросить урон дважды и выбрать желаемый результат.
	\end{itemize}

	\partc[Обычный материал]{Коралл}
	
	\noindent Обычно используется морскими существами для украшения оружия и брони. Это твердый материал, дарующий бонусы в воде.
	
	\begin{itemize}
		\item \textbf{Цена за единицу:} 100 зм
		
		\item \textbf{Доспех:} Вы получаете скорость плавания, равную вашей скорости ходьбы.
		
		\item \textbf{Оружие:} Все атаки, совершаемые в ближнем бою оружием из кораллов, не получают помехи при бросках атаки под водой.
	\end{itemize}

	\partc[Необычный материал]{Темная сталь}
	
	\noindent Металл, чей цвет зачастую сравнивают с темным ночным небом, который можно найти только в местах, где случаются частые грозы, бури и штормы. Известен тем, что привносит с собой энергию бури.
	
	\begin{itemize}
		\item \textbf{Цена за единицу:} 250 зм
		
		\item \textbf{Доспех:} урон от молний, который вы получаете от внешних источников, уменьшается на 1d10.
		
		\item \textbf{Оружие:} когда вы совершаете атаку по существу, вы наносите дополнительные 1d4 урона молнией.
	\end{itemize}

	\partc[Обычный материал]{Темная древесина}
	
	\noindent Эта древесина такая же прочная, как сталь, но невероятно легкая.
	
	\begin{itemize}
		\item \textbf{Цена за единицу:} 100 зм
		
		\item \textbf{Доспех:} убирает все ограничения доспеха по Силе.
	\end{itemize}

	\partc[Редкий материал]{Дварфийский камень}
	
	\noindent Этот камень, похожий на мрамор, зачастую используется дварфами при создании церемониальных доспехов несмотря на то, что они получаются тяжелее и неудобнее, чем металлические аналоги.
	
	\begin{itemize}
		\item \textbf{Цена за единицу:} 500 зм
		
		\item \textbf{Доспех:} Тяжелые доспехи дают дополнительные +1 к КД, но ограничения по Силе увеличиваются на +2.
	\end{itemize}

	\partc[Редкость варьируется]{Эллондийская шкура}
	
	\noindent Эту выгоревше-оранжевую шкуру получают с тел существ, живущих в засушливых областях, например в пустынях.
	
	\begin{itemize}
		\item \textbf{Цена за единицу:} определяется уровнем опасности существа.
		
		\item \textbf{Доспех:} вы можете без особых проблем переносить экстремальную жару. Вы не обязаны проходить проверки Телосложения, если вы находитесь в зоне с температурой до 38 $^\circ$C
	\end{itemize}

	\partc[Редкий материал]{Вечный лед}
	
	\noindent Этот необычайно твердый кусок небесно голубого льда переносит высокие температуры, благодаря чему его нельзя растопить при обычных условиях.
	
	\begin{itemize}
		\item \textbf{Цена за единицу:} 500 зм
		
		\item \textbf{Доспех:} урон холодом, который вы получаете из внешних источников, уменьшается на 1d10.
		
		\item \textbf{Оружие:} когда вы совершаете атаку по существу, вы наносите дополнительные 1d4 урона холодом.
	\end{itemize}

	\partc[Редкий материал]{Игнум}
	
	\noindent Этот черный, полупрозрачный камень, внутри которого находится что-то, напоминающее лаву, нагревающее его поверхность.
	
	\begin{itemize}
		\item \textbf{Цена за единицу:} 500 зм
		
		\item \textbf{Доспех:} урон огнем, который вы получаете из внешних источников, уменьшается на 1d10.
		
		\item \textbf{Оружие:} когда вы совершаете атаку по существу, вы наносите дополнительные 1d4 урона огнем.
	\end{itemize}

	\partc[Очень редкий материал]{Инфернальная кожа}
	
	\noindent Красноватая прочная кожа, созданная из шкуры Адской твари.
	
	\begin{itemize}
		\item \textbf{Цена за единицу:} 750 зм
		
		\item \textbf{Доспех:} вы получаете бонус +1 к КД и сопротивление к урону огнем.
	\end{itemize}

	\partc[Очень редкий материал]{Инфернальная сталь}
	
	\noindent Кроваво красный металл, который можно найти в Девяти Преисподних и который излучает тепло и ненависть в равных пропорциях.
	
	\begin{itemize}
		\item \textbf{Цена за единицу:} 750 зм
		
		\item \textbf{Доспех:} Вы получаете +1 бонус к КД и сопротивление к урону огнем.
		
		\item \textbf{Оружие:} когда вы совершаете атаку по существу, вы наносите дополнительные 2d4 урона огнем.
	\end{itemize}

	\partc[Обычный материал]{Плетеные листья}
	
	\noindent Узор этих от природы сплетенных листьев делает их необычайно прочными.
	
	\begin{itemize}
		\item \textbf{Цена за единицу:} 100 зм
		
		\item \textbf{Доспех:} вы получаете преимущество на проверки Ловкости (Скрытность) в лесах и подобных им областях.
	\end{itemize}

	\partc[Необычный материал]{Мифрил}
	
	\noindent Этот бело-серебристый материал легче и гибче обычной стали.
	
	\begin{itemize}
		\item \textbf{Цена за единицу:} 250 зм
		
		\item \textbf{Доспех:} Если доспех обычно дает помеху на проверки Ловкости (Скрытность) или имеет ограничение по Силе, то его мифрильная версия лишена этих недостатков.
		
		\item \textbf{Оружие:} Двуручное оружие, сделанное из мифрила, теряют свойство \textbf{тяжелое}, а все остальное оружие получает свойство \textbf{легкое}.
	\end{itemize}

	\partc[Редкость варьируется]{Чешуя монстра}
	
	\noindent Чешуйки могут быть различных цветов, форм и размеров. Некоторые из них могут даже совершать в себе остатки магии, которой до этого владело существо, у которого их забрали.
	
	\begin{itemize}
		\item \textbf{Цена за единицу:} определяется уровнем опасности существа.
		
		\item \textbf{Доспех:} Дробящий, колющий и рубящий урон, который вы получаете из внешних источников, уменьшается на 1d10. Возможны дополнительные эффекты, определяемые вашим Мастером. См. страницу 11 этого руководства для возможных примеров. % Пофиксить страницу
	\end{itemize}

	\partc[Необычный материал]{Обсидиан}
	
	\noindent Этот камень, с переливами от черного к темно-фиолетовому, похож на стекло и сверкает на свету. Известен за свою необычайную остроту.
	
	\begin{itemize}
		\item \textbf{Цена за единицу:} 250 зм
		
		\item \textbf{Доспех:} когда враг пытается взять вас в \textbf{захват}, он получает 1d4 рубящего урона.
		
		\item \textbf{Оружие:} Любое колющее или рубящее оружие, сделанное из обсидиана, наносит дополнительные 1d4 урона того же типа. 
	\end{itemize}

	\partc[Необычный материал]{Орихалк}
	
	\noindent Этот бронзово-оранжевый металл светится теплым светом вблизи источников магии. Используется для поглощения магической сущности.
	
	\begin{itemize}
		\item \textbf{Цена за единицу:} 500 зм
		
		\item \textbf{Доспех:} вы получаете преимущества на все спасброски против заклинаний и прочих магических эффектов.
		
		\item \textbf{Оружие:} когда вы совершаете атаку по существу, вы наносите дополнительные 1d4 урона силовым полем. 
	\end{itemize}

	\partc[Необычный материал]{Ядовитый Коа}
	
	\noindent Это покрытая мхом прочная древесина, вокруг которой парят мелкие споры.
	
	\begin{itemize}
		\item \textbf{Цена за единицу:} 250 зм
		
		\item \textbf{Доспех:} урон ядом, который вы получаете из внешних источников, уменьшается на 1d10.
		
		\item \textbf{Оружие:} когда вы совершаете атаку по существу, вы наносите дополнительные 1d4 урона ядом.
	\end{itemize}

	\partc[Редкость варьируется]{Теневой шелк}
	
	\noindent Этот полупрозрачный шелк бережно создается пауками и подобными им существами.
	
	\begin{itemize}
		\item \textbf{Цена за единицу:} определяется уровнем опасности существа.
		
		\item \textbf{Доспех:} вы получаете преимущество на проверки Ловкости (Скрытность) в слабо затемненных областях.
	\end{itemize}

	\partc[Очень редкий материал]{Шаед}
	
	\noindent Невероятно мягкий материал, который можно найти только в Царстве Теней, пронизанный темной сущностью плана.
	
	\begin{itemize}
		\item \textbf{Цена за единицу:} 750 зм
		
		\item \textbf{Доспех:} вы получаете бонус +1 к КД и сопротивление психическому урону.
	\end{itemize}

	\partc[Необычный материал]{Пятнистая береза}
	
	\noindent Пятна на этой крепкой древесину на самом деле являются небольшими отверстиями, покрывающими ее поверхность. Эти отверстия излучают едкий, жгучий запах.
	
	\begin{itemize}
		\item \textbf{Цена за единицу:} 250 зм
		
		\item \textbf{Доспех:} урон кислотой, который вы получаете от внешних источников, уменьшается на 1d10.
		
		\item \textbf{Оружие:} когда вы совершаете атаку по существу, вы наносите дополнительные 1d4 урона кислотой.
	\end{itemize}

	\partc[Очень редкий материал]{Небесное железо}
	
	\noindent Это ярко-белый металл с мягкой текстурой, но потрясающей воображение прочностью. Известно, что он наделен небесной силой.
	
	\begin{itemize}
		\item \textbf{Цена за единицу:} 750 зм
		
		\item \textbf{Доспех:} вы получаете бонус +1 к КД и сопротивление некротическому и сияющему урону. %Сияющему? Шура?
		
		\item \textbf{Оружие:} когда вы совершаете атаку по существу, вы наносите дополнительные 2d4 урона сиянием.
	\end{itemize}

	\chapter{Полевое руководство по флоре}
	
	Здесь вы найдете описание различных ингредиентов, которые собирают травники. Некоторые растения достаточно известны, чтобы их можно было тут же распознать, другие же могут оставаться неизвестными годами. Каждый ингредиент привносит уникальные свойства и применения для повернутого на алхимии участника вашей группы.
	
	\partc{Арктический плющ}
	
	\noindent  Это ядовитое растение обычно растёт в экстремально холодной окружающей среде, либо на огромной высоте, где снег покрывает вершины. Листья растения источают приятный сладковатый мятный аромат, а корень горький и содержит кислоту. Это растение очень популярно среди убийц, благодаря способности его корней замораживать кровообращение живых существ, что приводит к смерти в агонии. Арктический плющ зачастую оказывается смертельным для неподготовленных путешественников, которые с легкостью получают летальную дозу яда, наслаждаясь ароматом его листьев.
	
	\partc{Стрельчатый корень}
	
	\noindent Это необычно длинное растение может достигать четырёх футов в высоту. Его очень легко заметить из-за отличительного бело-коричневого окраса. Растёт в пустынях и других засушливых средах из-за низкой необходимости в воде. При нарезании кубиками и кипячении в воде, получается пенистый отвар серебристого цвета, который идеально подходит для заточки и \textbf{полировки оружия и брони}, без использования магии или других средств.
	
	\partc{Шляпка мухомора}
	
	\noindent Этот большой гриб часто встречается в грибницах вблизи водоёмов или в других влажных местах. У него толстый синий стебель с большой красной шапкой, что позволяет его довольно легко опознать. Профессиональные травники часто срезают только головку, потому что гриб имеет поразительную способность отращивать её заново всего за несколько недель.
	
	\partc{Дыхание василиска}
	
	\noindent Часто называемая ''Серым воздержанием'' дворянами всего мира, это тёмно-серая виноградная лоза, которая редко встречается на вершинах гор. Существует легенда, что эта лоза --- подарок богов, способ проверить человечество. Часто продаваемое за возмутительные суммы золота, Дыхание василиска легко может привлечь нежелательное внимание к тем, кто пытается продать её.
	
	\partc{Кровьтрава}
	
	\noindent Самое простое и обыкновенное растение, которое можно найти в дикой местности --- эта тёмно-коричневая трава. У него нет абсолютно никаких значительных качеств, кроме как относительно безвредных. При должной сноровке и умении она может стать отличным средством пропитания. Травники не считают эту траву чем-то особенным, но всё же обычно ее собирают, поскольку она практически не занимает места в сумках.
	
	\partc{Синий кровожаб}
	
	\noindent Этот гриб очень распространен. У него тёмно-синяя шляпка и жёлтый полосатый стеблем. Если до него дотронуться, то он выпустит облако синей пыльцы. Она не наносит никакого вреда живым существам, однако легко может ненадолго окрасить кожу или вещи. Этот порошок обычно используется для красок и красителей. Травники ищут этот гриб вокруг небольших водопоев, где живут водные обитатели.
	
	\partc{Сок кактуса}
	
	\noindent Прозрачная жидкость, может быть найдена в большинстве кактусов по всему миру. Её достаточно сложно извлечь в силу того, что со многими кактусами работать опасно. Пивовары любят использовать этот сок во многих рецептах, так как он отсрочивает опьянение и позволяет людям выпить больше, прежде чем их вырубит.
	
	\partc{Хромовая слизь}
	
	\noindent Эта тонкая, слизеподобная субстанция может быть найдена плавающей на поверхности воды, как будто она имеет собственный разум. Часто учёные ошибочно путают это вещество с ртутью из-за внешней схожести и консистенции. При попытке собрать слизь, она изменит и перестроит любую флору, взаимодействующую с ней.
	
	\partc{Небесный Глонд}
	
	\noindent Это четырехлистное растение до безобразия тяжело найти. Во многом это связано с тем, что растет оно в наиболее скрытых и непредсказуемых местах. Его листья по форме похожи на звезды ночного неба.
	
	\partc{Дьявольский кроволист}
	
	\noindent Существует лишь несколько подтверждений существования этих красно-жёлтых цветов. Это растение с большими красными листьями может быть найдено вновь только на закате человечества. А когда-то оно было популярным декором в множествах садов и домов. Говорят, что оно даёт огромную жизненную силу и здоровье тому, кто сможет правильно его приготовить.
	
	\partc{Цветок Дракуса}
	
	\noindent Эти ярко-красные и бледно-зеленые цветы могут быть найдены как в умеренном, так и в теплом климате. Это любимое растение артистов и фокусников из-за способности его лепестков самовозгораться при умеренном трении. Такой огонь не причиняет вреда, только создает красивые театральные искры.
	
	\partc{Сушеная Эфедра}
	
	\noindent Это шипастый куст, который растет преимущественно в сухом климате и который тяжело собрать, не поцарапав кожу. У него глубокий, темно-фиолетовый цвет издалека, которые превращается в черный вблизи. Травники зачастую находят сладковато-розовый запах притягательным. Его также можно использовать в лечебных отварах, поскольку он усиливает свойства Дикого шалфея.
	
	\partc{Рвоск}
	
	\noindent Этот плотный белый воск зачастую просачивается из деревьев во влажных областях. Его обычно используют при создании свечей, поскольку он быстро тает и вновь затвердевает и при этом достаточно прочный, чтобы из него можно было сделать что-то красивое. Несмотря на его полезность, длительное взаимодействие с ним может вызвать раздражение на коже.
	
	\partc{Фенхелевый шелк}
	
	Это белое растение, которое похоже на паутину, из-за чего его часто с ней путают, растет в холодных и темных местах. У него есть крочкообразные усики, которыми он закрепляется на ближайших камнях или растениях. Приключенцы, умеющие им пользоваться, могут придумать ему множество применений. Помимо очевидных, он может помочь в защите от низких температур северных гор.
	
	\partc{Дьявольский плющ}
	
	\noindent Эта длинная, шипастая лоза может достигать трех футов в длину, а шипы ее достигают дюйма в длину. Нередко на них можно увидеть кровь, так как многие животные и существа случайно натыкаются на плющ. Лоза кажется даже разумной, так как она расслабляется когда пуста и сжимается, когда в неё что-то попало.
	
	\partc{Замороженные саженцы}
	
	\noindent Эти небольшие стручки размером с горошину можно найти среди цветов в очень холодных условиях. Названные из-за своего замороженного вида, их достаточно часто собирают и используют в холодных алкогольных напитках.
	
	\partc{Листья Харрады}
	
	\noindent Эти большие желтые листья можно зачастую найти на самых верхушках крон деревьев.  Его часто культивируют и собирают различные банды и гильдии воров для изготовления и продажи наркотиков. Употребление вызывает кратковременную эйфорию, за которой следует длительный период восстановления. Местные власти борются с этим наркотиком, потому что слишком большие дозы могут причинить значительный вред, а иногда даже смерть.
	
	\begin{dndtable}
		
		{\Large \textbf{Наркотическая зависимость}}
		
		\noindent По решению Мастера, в игру может быть введена зависимость от некоторых веществ. Такие растения и ингредиенты, которые зачастую значительно увеличивают возможности персонажа, обычно оставляют свои следы в долгосрочной перспективе. Они могут быть найдены на стенках желудка или в легких спустя дни и недели после использования. Персонажи, употребляющие такие вещества, должны представлять себе последствия, которые те привнесут.
		
	\end{dndtable}
	
	
	
	\partc{Нектар Гиацинта}
	
	\noindent Эта бело-синяя густая жидкость может быть извлечена из гиацинта во влажном климате. Этот нектар пользуется большим спросом и часто используется высококвалифицированными охранниками для борьбы с ядами. Он не выводит яд полностью, но сильно ослабляет его свойства.
	
	\partc{Водополох}
	
	\noindent Названное в честь своей внешней схожести с известным растением, это трезубчатое сине-черное растение можно найти в тёмных и сырых средах обитания. Сам по себе не имеет особых положительных эффектов. Тем не менее, опытные алхимики смогли раскрыть его потенциал для создания зелий, позволяющих дышать под водой.
	
	\partc{Сердце железного дерева}
	
	\noindent Это маленькое белое семя обычно встречается в недрах железных деревьев. Эти семена медленно пульсируют, если их крепко сжать, часто их называют ''сердцебиение природы''. Говорят, что если его приготовит хороший травник, то оно позволит может увеличить физический размер существа.
	
	\partc{Побег лаванды}
	
	\noindent Эти фиолетовые цветы на длинном стебле зачастую можно найти качающимися на ветру в больших скоплениях. Они очень распространены на равнинах, у них сладких аромат, однако они довольно горькие на вкус.

	\partc{Светящаяся пыль}
	
	\noindent Этот порошок можно стрясти с желтых светящихся грибов, которые нередко можно найти в особенно темных местах. Он сохраняет янтарно-золотистое свечение на протяжении недели после сбора. Многие травники разводят светящиеся грибы в темных подвалах, чтобы всегда иметь доступ к этой пыли под рукой.
	
	\partc{Корень мандрагоры}
	
	\noindent  У этого коричневатого корня заостренные края по всей поверхности, из-за чего у многих травников, собирающих его неправильно, появляются раны. Если снять внешнюю оболочку то внутри окажется мягкая, нежная сердцевина, которую относительно просто можно съесть. Ее нередко используют врачи для уменьшения боли от яда или болезни.
	
	\partc{Семена молочая}
	
	\noindent Это маленькие белые полупрозрачные семена можно найти на цветах молочая. Их часто едят дети из-за их безобидного вида,что затем приводит к плохим последствия для пищеварения. Если эти семена перемолоть и разбавить жидкостью, то они сильно увеличат сопротивляемость всевозможным эффектам.
	
	\partc{Порошок умершей плоти} % ыы
	
	\noindent Этот тёмно-пурпурный порошок часто встречается на цветах мха в тёмных и холодных условиях. Этот порошок часто используется в качестве макияжа среди молодых мужчин и женщин для омоложения лица. При употреблении с магическим катализатором эффект становится постоянным.
	
	\partc{Ягоды паслена}
	
	\noindent Эти голубые ягоды можно встретить в маленьких скоплениях, среди кустов в богатой на растительность местности. Их можно есть без вреда для здоровья, животные часто едят их из-за их сладкого, но резковатого вкуса. Искусный травник может собрать их и раскрыть потенциал маслянистой мякоти.
	
	\partc{Изначальный бальзам}
	
	\noindent Было обнаружено, что это густое вещество меняет свою окраску будто бы по своему усмотрению. Бальзам необычно тёплый на ощупь и кажется, что он сохраняет тепло в течении нескольких недель подряд. Травники часто находят это вещество на скалах во влажных местах. Точная редкость вещества неизвестна, так как изменчивость его внешнего вида затрудняет его поиски.
	
	\partc{Ртутный лишайник}
	
	\noindent Этот серебристо-серый мох можно найти практически на чем угодно, создается впечатление, что он игнорирует вообще все известные правила. Из-за его мягкой, шелковистой текстуры его используют некоторые небольшие млекопитающие, чтобы устелить свои гнезда. Если его поджечь, то он начнет источать кислые ядовитые пары.
	
	\partc{Сияющий синтоцвет}
	
	\noindent Это длинное вытянутое чёрное семя испускает жёлтое свечение. Если сломать стручок, то он начнет источать цветочный запах. Внутри находится несколько более мелких семян, которые  которые светятся теплым золотистым светом. Если их размолоть, то они темнеют и превращаются в фиолетовую смолянистую субстанцию.
	
	\partc{Каменный вьюн}
	
	\noindent Эту чрезвычайно твёрдая тёмно-зелёная лоза можно найти на земле вблизи залежей минералов. Кажется, что она берет питательные вещества напрямую из минералов. Долгие годы эта лоза казалось бесполезной, однако магические исследования показали, что эта лоза --- важный компонент в защитных заклинаниях.
	
	\partc{Бобы Сцилли}
	
	\noindent Эти светло-коричневые бобы можно случайно найти висящими на кустах Сцилли в сухой местности. Их часто используют для улучшения вкуса при тушении мяса и приготовлении другой пищи. Если раскрыть их потенциал, то они позволят с легкостью карабкаться по скалам и горам.
	
	\partc{Серебряный гибискус}
	
	\noindent Это серо-серебряное растение выглядит так, как будто оно представляет необузданные элементальный потенциал. Зачастую оно растет круглыми симметричными узорами, которые сравнивают с паутиной, сплетенной на поверхности листьев. Говорят, что во время бурь листья сверкают одновременно с ударами молнии.
	
	\partc{Ягоды шипоцвета}
	
	\noindent Эти белые ягоды, которые часто можно найти среди костеобразных цветов, необходимо собирать аккуратно, чтобы не нарушить хрупкую структуру. Если их проглотить, то они немного утолят голод и вызовут небольшое чувство головокружения.
	
	\partc{Листохвост}
	
	\noindent Это очень странный тёмно-зелёный лист похож на круг с тремя толстыми нитями, которые с него свисают. При прикосновении к листу кажется, будто он вибрирует нераскрытой энергией.
	
	\partc{Зеленая крапива}
	
	\noindent Это растение легко определить из-за его зелено-желтых крапчатых листьев, чем-то напоминающих сетку. Обычно она растет в лесах. Неосторожный путешественник может случайно на нее наступить, если не будет смотреть под ноги. Рейнджеры говорят, что ее можно использовать для того, чтобы успокоить хищников.
	
	\partc{Корень пустоты}
	
	\noindent Этот темно-серый корень растет в самый суровых условиях, в пустынях и в арктике. Его темпы роста, кажется, варьируются от вида к виду. Травники очень аккуратно собирают корень, потому что создается впечатление, что он влияет на гравитацию вокруг себя.
	
	\partc{Корень дикого шалфея}
	
	\noindent Этот нежно-розовый корень --- дин из самых распространенных ингредиентов среди врачей и лекарей. Обычно они от трех до пяти дюймов в длину, у них мягкая, пушистая текстура. Мастера алхимии и лекари используют его для создания зелий необычайной лечебной силы.
	
	\partc{Стебли гифломы}
	
	\noindent Эти необычайно редкие грибы стали чем-то вроде сказки среди травников. Сообщается, что у них большая, похожая на луковицу шляпка, расположенная на тонкой ножке. Обычно они формирую скопления глубоко внутри влажных пещер и лесов. Их тело полупрозрачно-синее, а с некоторых углов оно кажется более тусклым, чем с других.
	
	\partc{Зловонная луковица}
	
	\noindent Эти большие белые луковицы могут быть найдены на болотах, в сырых пещерах и на гниющем остове корабля. Эти луковицы при прикосновении выпускают облако порошка, чтобы сбить с толку и запутать существ вокруг. Если же их успешно собрать, то их можно перемолоть в кашицу. Удивительно, но алхимики заявляют, что итоговый продукт совершенно не похож на то, с чем они начинали работать.
	
	\partc{Лепестки змееуста}
	
	\noindent Эти заостренные красные лепестки можно найти растущими на цветах змееуста практически всюду. На ветру лепестки поблескивают словно масло. Больше всего их любят паладины и те, кто ищут справедливости.
	
	
	\chapter{Опциональные правила травничества и алхимии}
	
	Как ДМ, вы можете решить использовать одно или несколько из приведенных ниже дополнительных правил в вашей игре, если будете пользоваться этим руководством. Однако обязательно обсудите это со своими игроками, чтобы они были в курсе событий.
	
	\section{Срок годности ингредиентов}
	
	Это опциональное правило травничества влияет на то, как долго будут храниться ингредиенты до тех пор, как их используют в качестве алхимического компонента или употребят каким-то иным способом. Если ингредиент был использован в алхимическом предмете, то он более не считается ингредиентом при расчете его срока годности.
	
	Большинство ингредиентов не портятся около недели игрового времени, после чего постепенно начинают терять потенциал. Если их использовать после этого времени, то у зелья может либо быть какой-то немного иной эффект, либо же не может быть эффекта вовсе. Тем не менее, редкие и очень редкие ингредиенты не портятся до месяца игрового времени из-за суровых условий, в которых они росли.
	
	Хорошим способом сохранить ингредиенты от ранней порчи может быть использование сумки для ингредиентов (25 зм, 1 фунт). Сумки могут быть различными, но стоят одинаково. Некоторые из них могут держать ингредиенты в сухости, некоторые --- убирать большую часть кислорода. Подходящая сумка \textbf{удваивает} время хранения ингредиента.
	
	\section{Токсичность зелий}
	
	Употребление слишком большого количества зелий может привести к самым разным последствиям, если не придавать этому значительного внимания. Если вы чувствуете, что ваши игроки используют слишком много зелий, это вариативное правило может помочь вам улучшить ситуацию: 
	
	\href{https://homebrewery.naturalcrit.com/share/n1MAVZSD}{Potion Quick Drinking and Toxicity for 5e}	
	
	\section{Сокровища Подземья}
	
	Немногие ищут сражения в Подземье и ищут там ингредиенты, чтобы сотворить невероятные смеси. С этим опциональным правилом, любые ингредиенты из Подземья и любые предметы, которые из них сделаны, имеют темное свечение вокруг них. Покупатели готовы купить их по пятикратной по сравнению с обычной цене. В дополнение к этому, вы можете решить, что у зелий или ядов могут быть дополнительные эффекты, связанные с Подземьем.
	
	\section{Опциональные запреты}
	
	\subsubsection{Ампулы}
	
	Ампулы содержат приготовленные яды и зелья. Это цилиндры диаметром 1 дюйм и высотой 3 дюйма, но одном конце которого есть герметичный клапан. Они сделаны из магического, неразрушимого стекла и могут включать до 40 мл жидкости. Каждое зелье или яд необходимо использовать полностью, чтобы получить его эффект.
	
	\subsubsection{Мешочек с ампулами}
	
	Эти мешочки могут вместить до \textbf{пяти ампул}. Их можно носить на поясе, патронташе или же его можно вшить в куртку или плащ. Они достаточно крепкие, чтобы в них можно было безопасно хранить до пяти ампул вне зависимости от того, что будет делать их носитель во время приключений.
	
	\subsubsection{Мешочек с ингредиентами}
	
	Мешочек, в котором можно хранить определенное количество алхимических ингредиентов для дальнейшего использования в зельях и ядах. В этих мешочках много разных отделений, чтобы держать ингредиенты раздельно, чтобы минимизировать вероятность какого-либо их ненарочного смешения.
	
	\onecolumn

	\chapter{Приложение А: списки ингредиентов}
	
	\begin{table}[H]
		
		{\Large \textbf{Растительные ингредиенты ядов}}
		
		\centering
		
		\medspace
		
		\begin{tabularx}{\linewidth}{X C c X}
			
			\textbf{Растение} & \textbf{Редкость} & \textbf{Мод-р КС} & \textbf{Биом} \\
			
			\cellcolor{altertable}Арктический плющ & \cellcolor{altertable}Обычный & \cellcolor{altertable}+2 & \cellcolor{altertable}Арктика, Горы \\
			
			Шляпка мухомора & Обычный & +1 & Побережье, Болота \\
			
			\cellcolor{altertable}Дыхание василиска & \cellcolor{altertable}Очень редкий & \cellcolor{altertable}+5 & \cellcolor{altertable}Горы \\
			
			Сок кактуса & Обычный & +2 & Пустыня, Равнина \\
			
			\cellcolor{altertable}Хромовая слизь & \cellcolor{altertable}Редкий & \cellcolor{altertable}+4 & \cellcolor{altertable}Побережье, Подземье \\
			
			Цветок Дракуса & Обычный & +2 & Пустыня, Равнина, Горы \\
			
			\cellcolor{altertable}Рвоск & \cellcolor{altertable}Обычный & \cellcolor{altertable}+1 & \cellcolor{altertable}Лес, Болота \\
			
			Замороженные саженцы & Редкий & +4 & Арктика, Горы \\
			
			\cellcolor{altertable}\textbf{Куст Генко} & \cellcolor{altertable}Необычный & \cellcolor{altertable}+2 & \cellcolor{altertable}Холмы, Подземье \\
			
			Листья Харрады & Обычный & +1 & Лес \\
			
			\cellcolor{altertable}Побег лаванды & \cellcolor{altertable}Обычный & \cellcolor{altertable}$-2$ & \cellcolor{altertable}Побережье, Равнина, Холмы \\
			
			Ртутный лишайник & Необычный & +3 & Большинство мест \\
			
			\cellcolor{altertable}Сияющий синтоцвет & \cellcolor{altertable}Редкий & \cellcolor{altertable}+4 & \cellcolor{altertable}Подземье \\
			
			Ягоды шипоцвета & Необычный & +3 & Пустыня, Болота \\
			
			\cellcolor{altertable}Лепестки змееуста & \cellcolor{altertable}Обычный & \cellcolor{altertable}0 & \cellcolor{altertable}Большинство мест \\
			
		\end{tabularx}
		
	\end{table}

	\begin{dndtable}
		
		{\Large \textbf{Большинство мест}}
		
		Фраза ''Большинство мест'' в таблице означает, что эти травы можно найти практически в любом месте, если вы решите искать именно их.
	\end{dndtable}

	\begin{table}[H]
		
		{\Large \textbf{Растительные ингредиенты зелий}}
		
		\medspace
		
		\centering 
		
		\begin{tabularx}{\linewidth}{X C c X}
			
			\textbf{Растение} & \textbf{Редкость} & \textbf{Мод-р КС} & \textbf{Биом} \\
			
			\cellcolor{altertable}Стрельчатый корень & \cellcolor{altertable}Необычный & \cellcolor{altertable}+2 & \cellcolor{altertable}Пустыня, Лес, Равнина \\
			
			Кровьтрава & Обычный & 0 & Большинство мест \\
			
			\cellcolor{altertable}Синий кровожаб & \cellcolor{altertable}Редкий & \cellcolor{altertable}+3 & \cellcolor{altertable}Побережье, Лес, Болото \\
			
			Хромовая слизь & Редкий & +4 & Побережье, Подземье \\
			
			\cellcolor{altertable}Небесный Глонд & \cellcolor{altertable}Редкий & \cellcolor{altertable}+3 & \cellcolor{altertable}Побережье, Пустыня \\
			
			Дьявольский кроволист & Очень редкий & +5 & Холмы, Болото, Подземье \\
			
			\cellcolor{altertable}Сушеная Эфедра & \cellcolor{altertable}Необычный & \cellcolor{altertable}+2 & \cellcolor{altertable}Пустыня, Горы \\
			
			Фенхелевый шелк & Обычный & +2 & Арктика, Подземье \\
			
			\cellcolor{altertable}Дьявольский плющ & \cellcolor{altertable}Редкий  & \cellcolor{altertable}+4 & \cellcolor{altertable}Холмы, Арктика, Подземье \\
			
			Нектар гиацинта & Обычный & +1 & Побережье, Равнина \\
			
			\cellcolor{altertable}Водополох & \cellcolor{altertable}Необычный & \cellcolor{altertable}+2 & \cellcolor{altertable}Побережье, Болото \\
			
			Сердце железного дерева & Необычный & +3 & Арктика, Лес, Холмы \\
			
			\cellcolor{altertable}Светящаяся пыль & \cellcolor{altertable}Редкий & \cellcolor{altertable}+4 & \cellcolor{altertable}Горы, Подземье \\
			
			Корень мандрагоры & Обычный & 0 & Большинство мест \\
			
			\cellcolor{altertable}Семена молочая & \cellcolor{altertable}Обычный & \cellcolor{altertable}+2 & \cellcolor{altertable}Большинство мест \\
			
			Порошок умершей плоти & Очень редкий & +5 & Арктика, Подземье \\
			
			\cellcolor{altertable}Ягоды паслена & \cellcolor{altertable}Необычный & \cellcolor{altertable}+3 & \cellcolor{altertable}Лес, Холмы \\
			
			Изначальный бальзам & Редкий & +4 & Горы, Болото, Подземье \\
			
			\cellcolor{altertable}Каменный вьюн & \cellcolor{altertable}Редкий & \cellcolor{altertable}+4 & \cellcolor{altertable}Холмы, Горы \\
			
			Бобы Сцилли & Обычный & +1 & Пустыня, Равнина \\
			
			\cellcolor{altertable}Серебряный гибискус & \cellcolor{altertable}Редкий & \cellcolor{altertable}+4 & \cellcolor{altertable}Арктика, Подземье \\
			
			Листохвост & Очень редкий & +5 & Равнина, Холмы \\
			
			\cellcolor{altertable}Зеленая крапива & \cellcolor{altertable}Необычный & \cellcolor{altertable}+2 & \cellcolor{altertable}Лес \\
			
			Корень пустоты & Очень редкий & +5 & Арктика, Пустыня \\
			
			\cellcolor{altertable}Корень дикого шалфея & \cellcolor{altertable}Обычный & \cellcolor{altertable}0 & \cellcolor{altertable}Большинство мест \\
			
			Стебли гифломы & Очень редкий & +5 & Лес, Подземье \\
			
			\cellcolor{altertable}Зловонная луковица & \cellcolor{altertable}Редкий & \cellcolor{altertable}+4 & \cellcolor{altertable}Побережье, Болото \\
			
		\end{tabularx}
	\end{table}

	\chapter{Приложение Б: формулы и создание ядов и зелий}
	
	\section{Яды}
	
	Яды можно изготовить, использовав 1 единицу растительного ингредиента \textbf{или} указанной части монстра. Эти части монстров не ограничиваются указанными в списке ниже, поскольку некоторые другие части монстров могут производить такой же или как минимум похожий эффект, как и указанные яды. В качестве альтернативы, указанные части монстров можно использовать как-то иначе.
	
	Если при изготовлении вы используете какое-то \textbf{дополнительное} подходящее растение или часть монстра, то время и стоимость  изготовления яда уменьшится вдвое. Помимо этого ряд ядов не требует изготовления, как, например, яд виверны, поскольку он извлекается напрямую из существа.
	
	Изготовления яда --- опасная работа. Если вы не владеете инструментами отравителя, то время и стоимость создания яда увеличивается на 50\%. Число дней, которое нетренированный персонаж проводит за изготовлением засчитывается в общее число дней, необходимое для получения владения инструментами отравителя (XGE стр. ). %Страница 
	
	\begin{table}[H]
		\begin{tabularx}{\linewidth}{Q W W X R}
			
			\textbf{Растение} & \textbf{Часть монстра} & \textbf{Тип} & \multicolumn{1}{c}{\textbf{Описание яда}} & \textbf{Стоимость} \\

			\cellcolor{altertable}Арктический ползун & \cellcolor{altertable}--- & \cellcolor{altertable}Контактный & \cellcolor{altertable}\textit{Масло таггита}. \textbf{КС 13 Телосложение} или стать отравленным на 24 часа. Отравленное существо получает статус \textbf{Бессознательный}. Эффект пропадает при получении урона (DMG 258). & \cellcolor{altertable}400 зм \\
			
			Шляпка мухомора & --- & Вдыхаемый & \textit{Эссенция эфира}. \textbf{КС 15 Телосложение} или стать отравленным на 8 часов. Отравленное существо получает статус \textbf{Бессознательный}. Эффект пропадает при получении урона или если его потрясти (DMG 258). & 300 зм \\
			
			\cellcolor{altertable}Дыхание василиска & \cellcolor{altertable}--- & \cellcolor{altertable}Вдыхаемый & \cellcolor{altertable}\textit{Дыхание василиска}. \textbf{КС 12 Телосложение}. Медленно парализует цель. При проваленном броске, существо \textbf{Опутано}. Повторите спасбросок в конце следующего хода. В случае успеха эффект оканчивается. При провале, тело существа перестает подчиняться ему и парализуется на 24 часа. & \cellcolor{altertable}12000 зм \\
			
			Сок кактуса & --- & Поглощаемый & \textit{Бледная настойка}. \textbf{КС 15 Телосложение} или получить 1d6 урона ядом и стать отравленным. Отравленное существо повторяет спасбросок каждые 24 часа, получая 1d6 урона ядом в случае провал. Пока яд активен, этот урон не может быть вылечен каким-либо способом. После 7 успешных спасбросков эффект оканчивается и существо может исцеляться нормально (DMG 258). & 250 зм \\
			
			\cellcolor{altertable}Хромовая слизь & \cellcolor{altertable}Яд виверны & \cellcolor{altertable}Оружейный & \cellcolor{altertable}\textit{Яд виверны}. \textbf{КС 15 Телосложение} или получить 7d6 урона ядом (половина урона при успешном спасброске) (DMG 258). & \cellcolor{altertable}1200 зм \\
			
			Цветок Дракуса & --- & Вдыхаемый & \textit{Злоба} \textbf{КС 15 Телосложение} или стать отравленным на 1 час. Отравленное существо \textbf{Ослеплено} (DMG 258). & 250 зм \\
			
			\end{tabularx}
		\end{table}
			
			
		\begin{table}[H]
			\begin{tabularx}{\linewidth}{Q W W X R}
				
			\textbf{Растение} & \textbf{Часть монстра} & \textbf{Тип} & \multicolumn{1}{c}{\textbf{Описание яда}} & \textbf{Стоимость} \\	
			
			\cellcolor{altertable}Рвоск & \cellcolor{altertable}Слизь ползающего падальщика & \cellcolor{altertable}Контактный & \cellcolor{altertable}\textit{Слизь ползающего падальщика}. \textbf{КС 13 Телосложение} или отравлен на 1 минуту. Отравленное сущетсво \textbf{Парализовано} и может совершать спасбросок в конце каждого хода для окончания эффекта (DMG 258). & \cellcolor{altertable}200 зм \\
			
			Замороженные саженцы & --- & Поглощаемый & \textit{Ночные слезы}. Не имеет эффекта до полуночи. Если до этого момента не нейтрализован, то \textbf{КС 17 Телосложение} или 9d6 урона ядом (половина урона при успешном спасброске) (DMG 258). & 1500 \\
			
			\cellcolor{altertable}\textbf{Куст Генко} & \cellcolor{altertable}--- & \cellcolor{altertable}Оружейный & \cellcolor{altertable}\textit{Яд Дроу}. \textbf{КС 13 Телосложение} или отравлен на 1 час. При провале больше чем на 5 существо \textbf{Бессознательно} на время отравления. Существо приходит в сознание при получении урона или если его потрясти. Изготавливается в темноте. & \cellcolor{altertable}200 зм \\
			
			Листья Харрады & Яд гигантской ядовитой змеи & Оружейный & \textit{Змеиный яд}. \textbf{КС 11 Телосложение} или получить3d6 урона ядом (половина урона при успешном спасброске) (DMG 258) & 200 зм \\
			
			\cellcolor{altertable}Побег лаванды & \cellcolor{altertable}--- & \cellcolor{altertable}Поглощаемый & \cellcolor{altertable}\textit{Кровь ассасина}. \textbf{КС 10 Телосложение} или 1d12 урона ядом и отравлен на 24 часа. При успешном спасброске половина урона и не отравлен (DMG 258). & \cellcolor{altertable}150 зм \\
			
			Ртутный лишайник & --- & Вдыхаемый & \textit{Дым жженого отура}. \textbf{КС 13 Телосложение} или 3d6 урона ядом. В начале каждого хода цель должна повторять спасбросок. За каждый последующий провал цель получает 1d6 урона ядом. После трех успешных бросков действие яда оканчивается (DMG 258). & 500 зм \\
			
			\cellcolor{altertable}Сияющий синтоцвет & \cellcolor{altertable}Яд лилового червя & \cellcolor{altertable}Оружейный & \cellcolor{altertable}\textit{Лиловая смерть}. \textbf{КС 19 Телосложение} или 12d6 урона ядом (половина урона при успешном спасброске) (DMG 258) & \cellcolor{altertable}2000 зм \\
			
			Ягоды шипоцвета & --- & Поглощаемый & \textit{Ступор}. \textbf{КС 15 Телосложение} или отравлен на 4d6 часов. Отравленное существо \textbf{Недееспособно} (DMG 258). & 600 зм \\
			
			\cellcolor{altertable}Лепестки змееуста & \cellcolor{altertable}--- & \cellcolor{altertable}Поглощаемый & \cellcolor{altertable}\textit{Сыворотка правды}. \textbf{КС 11 Телосложение} или отравлен на 1 час. Отравленное существо не может врать, как будто бы оно находится под заклинанием \textbf{Область истины} (DMG 258). & \cellcolor{altertable}150 зм \\
		\end{tabularx}
	\end{table}


	
		
	

% Кто такой этот ваш Куст Генко? В этой секции есть, в предыдущей нет. Дальше есть, в таблицах Подземья, например.

	\section{Зелья}
	
	
	
	Для создания зелий нужный \textbf{как} 1 единица указанного растительного ингредиента, \textbf{так и} указанная часть монстра, если в формуле не сказано обратное. Эти части монстров не ограничиваются указанными в списке ниже, поскольку некоторые другие части монстров могут производить такой же или как минимум похожий эффект, как и указанные яды. В качестве альтернативы, указанные части монстров можно использовать как-то иначе.
	
	Если при изготовлении вы используете какое-то \textbf{дополнительное} подходящее растение или часть монстра, то время и стоимость  изготовления яда уменьшится вдвое. 
	
	\begin{dndtable}
		{\Large \textbf{Мастерство зельеварения}}
		
		Если вы не владеете ни инструментами алхимика, ни \textbf{Интеллектом} (Магия), время и стоимость изготовления зелья увеличиваются на 50\%. Число дней, которое нетренированный персонаж проводит за изготовлением засчитывается в общее число дней, необходимое для получения владения инструментами алхимика (XGE стр. ). %Страница
	\end{dndtable}

	\begin{table}[H]
		
	\end{table}
	
	\begin{tabularx}{\linewidth}{T E X}
		
		\textbf{Растение} & \textbf{Часть монстра} &  \multicolumn{1}{c}{\textbf{Зелье}} \\
		
		\cellcolor{altertable}Стрельчатый корень & \cellcolor{altertable}--- & \cellcolor{altertable}Масло остроты с дополнительным бонусом +1 (DMG) \\
		
		Синий кровожаб & --- & Зелье газообразной формы (DMG) \\
		
		\cellcolor{altertable}Хромовая слизь & \cellcolor{altertable}--- &  \cellcolor{altertable}Масло эфирности (DMG) \\
		
		Небесный Глонд & \textit{Глаз второго взгляда} & Зелье ясновидения (DMG) \\
		
		\cellcolor{altertable}Дьявольский кроволист & \cellcolor{altertable}\textit{Восстановленная эссенция жизни} & \cellcolor{altertable}Зелье живучести (DMG) \\
		
		Сушеная Эфедра & --- & Любовное зелье (DMG) \\
		
		\cellcolor{altertable}Фенхелевый шелк & \cellcolor{altertable}--- & \cellcolor{altertable}Зелье больших шагов. На 1 час к вашей скорости передвижения добавляется 10 футов. Зелье постепенно переливается от ночного черного к золотисто-желтому, если его покрутить. \\
		
		Дьявольский плющ & --- & Зелье чтения мыслей (DMG) \\
		
		\cellcolor{altertable}Нектар гиацинта & \cellcolor{altertable}--- & \cellcolor{altertable}Зелье многих языков. На 1 час вы оказываетесь под действием заклинания ''Понимание языков'' (Comprehend languages). Цвет зелья меняется согласно цветам радуги. \\
		
		Водополох & --- & Зелье подводного дыхания (DMG) \\
		
		\cellcolor{altertable}Сердце железного дерева & \cellcolor{altertable}--- & \cellcolor{altertable}Зелье увеличения (DMG) \\
		
		Светящаяся пыль & --- & Зелье героизма \\
		
		\cellcolor{altertable}Корень мандрагоры & \cellcolor{altertable}--- & \cellcolor{altertable}Зелье ложной жизни. Вы получаете 1d4 + 4 временных ХП на 1 час. Зелье представляет собой иссиня-черную жидкость с белым вихрем, медленно вращающимся внутри.\\
		
		Семена молочая & \textit{Настойка сопротивления} &  Зелье сопротивления (DMG). Если зелье изготавливается без настойки, то тип урона выбирается случайно.\\
		
		\cellcolor{altertable}Порошок умершей плоти & \cellcolor{altertable}Хвост скорпиона, клык гадюки, паук и крошечное бьющееся сердце & \cellcolor{altertable}Зелье долголетия \\
		
		Ягоды паслена & --- & Масло ускользания  \\
		
		\cellcolor{altertable}Изначальный бальзам & \cellcolor{altertable}Осколок ногтя великана & \cellcolor{altertable}Зелье силы великана. Тип зелья соответствует типу великана, осколок ногтя которого был использован. \\
		
		Каменный вьюн & --- & Зелье неуязвимости \\
		
		\cellcolor{altertable}Бобы Сцилли & \cellcolor{altertable}--- & \cellcolor{altertable}Зелье лазания \\
		
		Серебряный гибискус & Чешуя дракона & Зелье драконьего дыхания. Работает так же, как \textbf{Зелье огненного дыхания}, но тип урона соответствует использованной при изготовлении чешуе. Если зелье изготавливалось без нее, то получается \textbf{Зелье огненного дыхания}.  \\
		
		\cellcolor{altertable}Листохвост & \cellcolor{altertable}\textit{Эссенция ловкости} & \cellcolor{altertable}Зелье скорости (DMG) \\
		
		Зеленая крапива & \textit{Эссенция дикости} & Зелье дружбы с животными \\
		
		\cellcolor{altertable}Корень пустоты & \cellcolor{altertable}\textit{Эссенция подавления гравитации} & \cellcolor{altertable}Зелье полета (DMG) \\
		
		Корень дикого шалфея &  & \textit{Зелье лечения} (затраченных единиц/тип зелья): 1/обычное, 4/большое, 40/отличное, 400/превосходное (XGE) \\ %Проверить XGE????
		
		\cellcolor{altertable}Стебли гифломы & \cellcolor{altertable}Эссенция невидимого & \cellcolor{altertable}Зелье невидимости (DMG) \\
		
		Зловонная луковица & --- & Зелье уменьшения \\
		
	\end{tabularx}

% Что насчет кровьтравы? В таблице регов есть, тут нет нигде. Можно какое-нибудь зелье сытости из нее создать, например.

	\twocolumn

	\chapter{Приложение В: таблицы мест произрастания растений} %подумай еще раз
	
	Таблицы, приведенные здесь, используются для случайного определения типа растения, найденного при проведении проверки \textbf{Сбора} навыка травничества.
	
	Рядом с некоторыми ингредиентами в таблице расположены дополнительные правила. Например, единицу или две водополоха можно найти под водой, однако большие его количества встречаются только на \textbf{Болоте} во время дождя.
	
	\begin{table}[H]
		
		{\Large \textbf{Распространенные ингредиенты}}
		
		\medspace 
		
		\centering 
		
		\begin{tabularx}{\linewidth}{c C}
			
			\textbf{1d12}& \textbf{Название ингредиента} \\
			
			\cellcolor{altertable}1 & \cellcolor{altertable}Корень мандрагоры \\
			
			2 & Корень мандрагоры \\
			
			\cellcolor{altertable}3 & \cellcolor{altertable}Ртутный лишайник \\
			
			4 & Ртутный лишайник  \\
			
			\cellcolor{altertable}5 & \cellcolor{altertable}Корень дикого шалфея  \\
			
			6 & Корень дикого шалфея \\
			
			\cellcolor{altertable}7 & \cellcolor{altertable}Корень дикого шалфея  \\
			
			8 & Лепестки змееуста \\
			
			\cellcolor{altertable}9 & \cellcolor{altertable}Лепестки змееуста  \\
			
			10 & Семена молочая \\
			
			\cellcolor{altertable}11 & \cellcolor{altertable}Семена молочая  \\
			
			12 & Кровьтрава  \\ % в оригинале написано про сухарь, но зачем, казалось бы, если пишем в зельях?
			
		\end{tabularx}
	\end{table}

	\begin{table}[H]
		
		{\Large \textbf{Равнина}}
		
		\medspace 
		
		\centering 
		
		\begin{tabularx}{\linewidth}{c C}
			
			\textbf{2d6} & \textbf{Название ингредиента} \\
			
			\cellcolor{altertable}2 & \cellcolor{altertable}Листья Харрады  \\
			
			3 & Цветок Дракуса  \\
			
			\cellcolor{altertable}4 & \cellcolor{altertable}Побег лаванды  \\
			
			5 & Стрельчатый корень  \\
			
			\cellcolor{altertable}6 & \cellcolor{altertable}Обычный ингредиент \\
			
			7 & Обычный ингредиент \\
			
			\cellcolor{altertable}8 & \cellcolor{altertable}Обычный ингредиент \\
			
			9 & 2x Бобы Сцилли \\
			
			\cellcolor{altertable}10 & \cellcolor{altertable}Сок кактуса  \\
			
			11 & Листохвост  \\
			
			\cellcolor{altertable}12 & \cellcolor{altertable}Нектар гиацинта
		\end{tabularx}
	\end{table}

	\begin{table}[H]
		
		{\Large \textbf{Холмы}}
		
		\medspace 
		
		\centering 
		
		\begin{tabularx}{\linewidth}{c C}
			
			\textbf{2d6} & \textbf{Название ингредиента} \\
			
			\cellcolor{altertable}2 & \cellcolor{altertable}Дьявольский кроволист  \\
			
			3 & Ягоды паслена  \\
			
			\cellcolor{altertable}4 & \cellcolor{altertable}2x Листохвост \\
			
			5 & Побег лаванды  \\
			
			\cellcolor{altertable}6 & \cellcolor{altertable}Обычный ингредиент \\
			
			7 & Обычный ингредиент \\
			
			\cellcolor{altertable}8 & \cellcolor{altertable}Обычный ингредиент \\
			
			9 & Сердце железного дерева  \\
			
			\cellcolor{altertable}10 & \cellcolor{altertable}Куст Генко  \\
			
			11 & Каменный вьюн  \\
			
			\cellcolor{altertable}12 & \cellcolor{altertable}Листья Харрады
		\end{tabularx}
	\end{table}

\begin{table}[H]
	
	{\Large \textbf{Лес}}
	
	\medspace 
	
	\centering 
	
	\begin{tabularx}{\linewidth}{c C}
		
		\textbf{2d6} & \textbf{Название ингредиента} \\
		
		\cellcolor{altertable}2 & \cellcolor{altertable}Листья Харрады \\
		
		3 & Ягоды паслена  \\
		
		\cellcolor{altertable}4 & \cellcolor{altertable}Рвоск \\
		
		5 & Зеленая крапива  \\
		
		\cellcolor{altertable}6 & \cellcolor{altertable}Обычный ингредиент \\
		
		7 & Обычный ингредиент \\
		
		\cellcolor{altertable}8 & \cellcolor{altertable}Обычный ингредиент \\
		
		9 & Стрельчатый корень  \\
		
		\cellcolor{altertable}10 & \cellcolor{altertable}Сердце железного дерева  \\
		
		11 & Синий кровожаб  \\
		
		\cellcolor{altertable}12 & Ночь: \cellcolor{altertable}2x Стебли гифломы/перебросить
	\end{tabularx}
\end{table}

	\begin{table}[H]
		
		{\Large \textbf{Пустыня}}
		
		\medspace 
		
		\centering 
		
		\begin{tabularx}{\linewidth}{c C}
			
			\textbf{2d6} & \textbf{Название ингредиента} \\
			
			\cellcolor{altertable}2 & \cellcolor{altertable}Небесный глонд  \\
			
			3 & Стрельчатый корень  \\
			
			\cellcolor{altertable}4 & \cellcolor{altertable}Сушеная Эфедра \\
			
			5 & 2x Сок кактуса \\
			
			\cellcolor{altertable}6 & \cellcolor{altertable}Обычный ингредиент \\
			
			7 & Обычный ингредиент \\
			
			\cellcolor{altertable}8 & \cellcolor{altertable}Обычный ингредиент \\
			
			9 & Цветок Дракуса  \\
			
			\cellcolor{altertable}10 & \cellcolor{altertable}Бобы Сцилли  \\
			
			11 & Ягоды шипоцвета  \\
			
			\cellcolor{altertable}12 & \cellcolor{altertable}Корень пустоты  \\
		\end{tabularx}
	\end{table}

	\begin{table}[H]
		
		{\Large \textbf{Водная гладь/Побережье}}
		
		\medspace 
		
		\centering 
		
		\begin{tabularx}{\linewidth}{c C}
			
			\textbf{2d6} & \textbf{Название ингредиента} \\
			
			\cellcolor{altertable}2 & \cellcolor{altertable}1/2x Водополох \\
			
			3 & Шляпка мухомора \\
			
			\cellcolor{altertable}4 & \cellcolor{altertable}Нектар гиацинта \\
			
			5 & Хромовая слизь \\
			
			\cellcolor{altertable}6 & \cellcolor{altertable}Обычный ингредиент \\
			
			7 & Обычный ингредиент \\
			
			\cellcolor{altertable}8 & \cellcolor{altertable}Обычный ингредиент \\
			
			9 & Побег лаванды (только на побережье) \\
			
			\cellcolor{altertable}10 & \cellcolor{altertable}Синий кровожаб  \\
			
			11 & Зловонная луковица  \\
			
			\cellcolor{altertable}12 & \cellcolor{altertable}1/2x Небесный Глонд \\
		\end{tabularx}
	\end{table}

	\begin{table}[H]
		
		{\Large \textbf{Горы}}
		
		\medspace 
		
		\centering 
		
		\begin{tabularx}{\linewidth}{c C}
			
			\textbf{2d6} & \textbf{Название ингредиента} \\
			
			\cellcolor{altertable}2 & \cellcolor{altertable}Дыхание василиска  \\
			
			3 & Замороженные саженцы  \\
			
			\cellcolor{altertable}4 & \cellcolor{altertable}Арктический плющ \\
			
			5 & Сушеная Эфедра  \\
			
			\cellcolor{altertable}6 & \cellcolor{altertable}Обычный ингредиент \\
			
			7 & Обычный ингредиент \\
			
			\cellcolor{altertable}8 & \cellcolor{altertable}Обычный ингредиент \\
			
			9 & Цветок Дракуса  \\
			
			\cellcolor{altertable}10 & \cellcolor{altertable}Светящаяся пыль (2x в \textbf{пещерах}) \\
			
			11 & Каменный вьюн  \\
			
			\cellcolor{altertable}12 & \cellcolor{altertable}Изначальный бальзам  \\
		\end{tabularx}
	\end{table}

	\begin{table}[H]
		
		{\Large \textbf{Болото}}
		
		\medspace 
		
		\centering 
		
		\begin{tabularx}{\linewidth}{c C}
			
			\textbf{2d6} & \textbf{Название ингредиента} \\
			
			\cellcolor{altertable}2 & \cellcolor{altertable}Дьявольский кроволист \\
			
			3 & Ягоды шипоцвета \\
			
			\cellcolor{altertable}4 & \cellcolor{altertable}Рвоск \\
			
			5 & 2x Шляпка мухомора \\
			
			\cellcolor{altertable}6 & \cellcolor{altertable}Обычный ингредиент \\
			
			7 & Обычный ингредиент \\
			
			\cellcolor{altertable}8 & \cellcolor{altertable}Обычный ингредиент \\
			
			9 & 2x Синий кровожаб \\
			
			\cellcolor{altertable}10 & \cellcolor{altertable}Зловонная луковица \\
			
			11 & Водополох (2x во время дождя)\\
			
			\cellcolor{altertable}12 & \cellcolor{altertable}Изначальный бальзам
		\end{tabularx}
	\end{table}

	\newpage

	\begin{table}[H]
		
		{\Large \textbf{Арктика}}
		
		\medspace 
		
		\centering 
		
		\begin{tabularx}{\linewidth}{c C}
			
			\textbf{2d6} & \textbf{Название ингредиента} \\
			
			\cellcolor{altertable}2 & \cellcolor{altertable}Серебряный гибискус \\
			
			3 & Порошок умершей плоти \\
			
			\cellcolor{altertable}4 & \cellcolor{altertable}Сердце железного дерева \\
			
			5 & 2x Замороженные саженцы \\
			
			\cellcolor{altertable}6 & \cellcolor{altertable}Обычный ингредиент \\
			
			7 & Обычный ингредиент \\
			
			\cellcolor{altertable}8 & \cellcolor{altertable}Обычный ингредиент \\
			
			9 & 2x Арктический плющ \\
			
			\cellcolor{altertable}10 & \cellcolor{altertable}Фенхелевый шелк \\
			
			11 & Дьявольский плющ \\
			
			\cellcolor{altertable}12 & \cellcolor{altertable}Корень пустоты
		\end{tabularx}
	\end{table}

	\begin{table}[H]
		
		{\Large \textbf{Подземье}}
		
		\medspace 
		
		\centering 
		
		\begin{tabularx}{\linewidth}{c C}
			
			\textbf{2d6} & \textbf{Название ингредиента} \\
			
			\cellcolor{altertable}2 & \cellcolor{altertable}1/2x Изначальный бальзам \\
			
			3 & Серебряный гибискус  \\
			
			\cellcolor{altertable}4 & \cellcolor{altertable}Дьявольский кроволист  \\
			
			5 & Хромовая слизь  \\
			
			\cellcolor{altertable}6 & \cellcolor{altertable}2x Порошок умершей плоти \\
			
			7 & Фенхелевый шелк \\
			
			\cellcolor{altertable}8 & \cellcolor{altertable}Дьявольский плющ \\
			
			9 & \textit{Куст Генко} \\
			
			\cellcolor{altertable}10 & \cellcolor{altertable}2x Светящаяся пыль \\
			
			11 & Сияющий синтоцвет \\
			
			\cellcolor{altertable}12 & \cellcolor{altertable}Стебли гифломы
		\end{tabularx}
	\end{table}
	
	
	
	
	
	
	
	
	
	
	
	
	
	
	
\end{document}